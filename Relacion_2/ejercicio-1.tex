\subproblem
Calculamos la tabla de frecuencia y además, los elementos que necesitaremos durante el ejercicio.

\begin{table}[h]
    \centering
    \begin{tabular}{|c|c|c|c|}
        \hline
         $x_i$ & $n_{i.}$ & $x_{i}n_{i.}$ & $x_i^2 n_{i.}$ \\ \hline
         1 & 4 & 4 & 4 \\ \hline 
         2 & 3 & 6 & 12 \\ \hline 
         3 & 5 & 15 & 45\\ \hline 
         4 & 6 & 24 & 96 \\ \hline 
         5 & 4 & 20 & 100 \\ \hline 
         6 & 2 & 12 & 72 \\ \hline 
           & 24 & 81 & 329 \\ \hline 
    \end{tabular}
\end{table}

\begin{table}[h]
    \centering
    \begin{tabular}{|c|c|c|c|}
        \hline
         $y_j$ & $n_{.j}$ & $y_{j}n_{.j}$ & $y_{j}^2n_{.j}$ \\ \hline
         1 & 6 & 6 & 6 \\ \hline 
         2 & 6 & 12 & 24 \\ \hline 
         3 & 2 & 6 & 18 \\ \hline 
         4 & 2 & 8 & 32 \\ \hline 
         5 & 3 & 15 & 75 \\ \hline 
         6 & 5 & 30 & 180 \\ \hline
           & 24 & 77 & 335 \\ \hline 
    \end{tabular}
\end{table}

\subproblem
Calculamos la media de nuestras variables:

\begin{equation*}
    \overline{x} = \dfrac{1}{n} \sum_{i=1}^k x_i n_{i.} = \dfrac{81}{24} = 3.375
    \hspace{1cm}
    \overline{y} = \dfrac{1}{n} \sum_{k=1}^p y_j n_{j.} = \dfrac{77}{24} = 3.2083
\end{equation*}

Ahora calculamos la desviación típica de cada variable para poder calcular el coeficiente de variación de Pearson y ver cuál es más homogénea:

\begin{equation*}
    \sigma_{x} = \sqrt{\dfrac{328}{24}-3.375^2} = 1.5224
    \hspace{1cm}
    \sigma_{y} = \sqrt{\dfrac{335}{24}-3.2083^2} = 1.9144
\end{equation*}

\begin{equation*}
    CV_{x} = \dfrac{\sigma_x}{\overline{x}} = 0.4511
    \hspace{1cm}
    CV_{y} = \dfrac{\sigma_y}{\overline{y}} = 0.5967
\end{equation*}

Como vemos, los resultados de la primera variable son más homogéneos pues su coeficiente de variación de Pearson es menor.

\subproblem
Mirando la tabla de resultados, vemos que cuando ha salido un 3 en el segundo dado, el número que más veces ha aparecido es el 4, con 2 veces, luego $M_0=4$.

\subproblem
Vamos a calcular la media de $X/Y$ cuando en el segundo dado se obtiene un 2 o un 5. Mirando la tabla de resultados obtenemos:

\begin{table}[ht]
    \begin{tabular}{|c|c|c|}
        \hline
         $x_i/Y$ & $n_{i}$ & $N_{i}$ \\ \hline
         1 & 1 & 1 \\ \hline 
         2 & 1 & 2 \\ \hline 
         3 & 1 & 3 \\ \hline 
         4 & 3 & 6 \\ \hline 
         5 & 1 & 7 \\ \hline  
         6 & 2 & 9 \\ \hline 
    \end{tabular}
\end{table}

Como $n/2 = 4.5$ y la frecuencia absoluta acumulada inmediatamente superior a este número es $N_{4} = 6$, deducimos que la mediana es $M_e = 4$.
