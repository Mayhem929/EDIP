\problem

En una encuesta de  familias sobre el n{\'u}mero  de  individuos
que  la componen ($X$) y el n{\'u}mero de personas activas en ellas
($Y$)  se  han obtenido los siguientes resultados:
$$
    \begin{array}{c|cccc}
          X \backslash Y &   1 &  2 &  3  & 4 \\ \hline
           1    &   7 &  0 &  0  & 0  \\
           2    &  10 &  2 &  0  & 0  \\
           3    &  11 &  5 &  1  & 0  \\
           4    &  10 &  6 &  6  & 0  \\
           5    &   8 &  6 &  4  & 2  \\
           6    &   1 &  2 &  3  & 1  \\
           7    &   1 &  0 &  0  & 1  \\
           8    &   0 &  0 &  1  & 1  \\
     \end{array}
$$

  \begin{enumerate}

     \item Calcular la recta de regresi\'on de $Y$ sobre $X$.
  \item
   ?`Es adecuado suponer una relaci\'on lineal para explicar el comportamiento de $Y$ a partir de $X$?
  \end{enumerate}

\subproblem
Esta será la tabla que utilicemos en el ejercicio, con todos los datos necesarios ya calculados.

\begin{table}[ht]
    \begin{tabular}{|c|c|c|c|c|c|c|c|c|c|}
        \hline
         $X/Y$ & 1 & 2 & 3 & 4 & $n_{i.}$ & $n_{i.} x_i$ & $n_{i.} x_i^2$ & $\sum_{j=1}^p n_{ij} y_j$ & $x_i \sum_{j=1}^p n_{ij} y_j$ \\ \hline
         1 & 7 & 0 & 0 & 0 & 7 & 7 & 7 & 7 & 7 \\ \hline 
         2 & 10 & 2 & 0 & 0 & 12 & 24 & 48 & 14 & 28 \\ \hline 
         3 & 11 & 5 & 1 & 0 & 17 & 51 & 153 & 24 & 72 \\ \hline 
         4 & 10 & 6 & 6 & 0 & 22 & 88 & 352 & 40 & 160 \\ \hline 
         5 & 8 & 6 & 4 & 2 & 20 & 100 & 500 & 40 & 200 \\ \hline 
         6 & 1 & 2 & 3 & 1 & 7 & 42 & 252 & 18 & 108 \\ \hline 
         7 & 1 & 0 & 0 & 1 & 2 & 14 & 98 & 5 & 35 \\ \hline
         8 & 0 & 0 & 1 & 1 & 2 & 16 & 128 & 7 & 56 \\ \hline
         $n_{.j}$ & 48 & 21 & 15 & 5 & 89 & 342 & 1538 &  & 666 \\ \hline
         $n_{.j}y_j$ & 48 & 42 & 45 & 20 & 155 &  &  &  &  \\ \hline
         $n_{.j}y_j^2$ & 48 & 84 & 135 & 80 & 347 &  &  &  &  \\ \hline
    \end{tabular}
\end{table}

Para calcular la recta de regresión de Y sobre X tenemos que tener en cuenta lo siguiente:

\begin{equation*}
    y = ax+b
    \hspace{1cm}
    a = \dfrac{\sigma_{xy}}{\sigma_x^2}
    \hspace{1cm}
    b = \overline{y} - a \overline{x}
    \hspace{1cm}
\end{equation*}

Ahora haremos los cálculos necesarios:

\begin{center}
    \begin{equation*}
        \overline{x} = \dfrac{1}{n} \sum_{i=1}^k x_i n_{i.} = \dfrac{342}{89} = 3.8427 \approx 4 personas
        \hspace{1cm}
        \overline{y} = \dfrac{1}{n} \sum_{k=1}^p y_j n_{j.} = \dfrac{155}{89} = 1.7415 \approx 2 personas\\
        
        \sigma_{x}^2 = m_{11} - m_{10}^2 = \dfrac{1539}{89} - \overline{x}^2 = 2.5146
        \hspace{1cm}
        \sigma_{xy} = m_{11} - m_{10}m_{01} = \dfrac{666}{89} - \overline{x}\overline{y} = 0.791
    \end{equation*}
\end{center}

Teniendo en cuenta los resultados obtenidos:

\begin{equation*}
    a = \dfrac{\sigma_{xy}}{\sigma_x^2} = 0.315
    \hspace{1cm}
    b = \overline{y} - a \overline{x} = 0.531
    \hspace{1cm}
\end{equation*}

Y, finalmente, nuestra recta de regresión es:

\begin{equation*}
    y = 0.315x + 0.531
\end{equation*}

\subproblem
Para saber si es adecuada o no esta suposición, calcularemos el coeficiente de determinación lineal. Para ello necesitamos calcular lo siguiente:

\begin{equation*}
    \sigma{y}^2 = m_{02} - m_{01}^2 = \dfrac{347}{89} - \overline{y}^2 = 0.866
\end{equation*}

Por último, calculamos el coeficiente:

\begin{equation*}
    r^2 = \dfrac{\sigma_{xy}^2}{\sigma_x^2 \sigma_y^2} = 0.287
\end{equation*}

Esto significa que la recta que hemos calculado, explica el 28.7\% de nuestra distribución, con lo que deducimos que no es adecuado suponer una relación lineal.
