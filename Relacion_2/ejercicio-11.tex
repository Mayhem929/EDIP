\subproblem
Calculamos la media de X con los datos que nos dan:

\begin{equation*}
    \overline{x} = \dfrac{1}{n} \sum_{i=1}^k n_{i.} x_i = \dfrac{45}{9} = 5
\end{equation*}

Además, como sabemos que se cumple que $\overline{y} = a \overline{x} + b$, obtenemos que:

\begin{equation*}
    \overline{y} = \overline{x} \cdot 5 - 20 = 5
\end{equation*}

Ahora empezamos a calcular la recta de Y sobre X. Para ello debemos tener en cuenta que:

\begin{equation*}
    x = a'y + b'
    \hspace{1cm}
    a' = \dfrac{\sigma_{xy}}{\sigma_y^2}
    \hspace{1cm}
    b' = \overline{x} - a' \overline{y}
    \hspace{1cm}
\end{equation*}

Ahora haremos los cálculos necesarios:

\begin{center}
    \begin{equation*}
            \sigma_{y}^2 = m_{11} - m_{10}^2 = \dfrac{3240}{9} - \overline{y}^2 = 335
            \hspace{1cm}
            \sigma_{x}^2 = m_{11} - m_{10}^2 = \dfrac{279}{9} - \overline{y}^2 = 6\\
            
            a = \dfrac{\sigma_{xy}}{\sigma_x^2} \Rightarrow \sigma_{xy} = 5 \cdot 6 = 30
    \end{equation*}
\end{center}

Terminamos de calcular la recta:

\begin{equation*}
    a' = \dfrac{\sigma_{xy}}{\sigma_y^2} = 0.0896
    \hspace{1cm}
    b' = \overline{x} - a' \overline{y} = 4.552
    \hspace{1cm}
\end{equation*}

Luego,

\begin{equation*}
    x = a'y + b' \Rightarrow x = 0.0896y + 4.552
\end{equation*}

Ahora veremos la bondad de los ajustes lineales. Para ello calculamos el coeficiente de determinación lineal:

\begin{equation*}
    r^2 = \dfrac{\sigma_{xy}^2}{\sigma_x^2 \sigma_y^2} = 0.448
\end{equation*}

Esto significa que la recta que hemos calculado explica el 44.8\% de nuestra distribución, por lo que podemos decir que no es bueno suponer una relación lineal.
