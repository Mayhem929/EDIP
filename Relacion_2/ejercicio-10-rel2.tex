\documentclass{article}

\begin{document}
	Ejercicio 10:
	
\begin{center}
	$x + 4y = 1 $\\
$x + 5y = 2$
\end{center}
	
	$\bullet$Supongamos que la primera ecuación es para X/Y y la segunda para Y/X:\\ \\
	$y = \frac{-1}{5}x + \frac{2}{5}$\\ \\
	$x = -4y + 1$\\ \\
	$\bullet$Las pendientes tienen que tener el mismo signo o sino no serían rectas de regresión, en este caso las dos pendientes son negativos.\\ 
	Sabemos que : \\
	
	
	$a = \frac{v_{xy}}{v_{x}^{2}} $ , $a^{'} = \frac{v_{xy}}{v_{y}^{2}} $\\ \\
	Luego :\\ \\
	$a.a^{'} = \frac{v_{xy}^{2}}{v_{x}^{2}v_{y}^{2}} = r^{2} = \frac{4}{5} $\\ \\
	Entonces esta bien porque sabemos que $ 0 <= r^{2} <= 1 $: \\ \\
	$r = \sqrt{ r^{2}}= \frac{-2}{\sqrt{5}} $
	
\end{document}