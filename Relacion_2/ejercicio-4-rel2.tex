\documentclass{article}

\begin{document}
		Ejercicio 4: \\ \\
	$\bullet$ En una población de tamaño n = 21 medidas se ha observado dos variables
	estadísticas, $X =$ temperatura en ºC e $Y =$ vapor de agua en ml de Hg, las
	cuales han presentado k = 3, p = 3 modalidades distintas, con distribución de
	frecuencia conjunta $(x_{i},y_{j})$, $n_{ij} i = 1,...,3$ $ j=1,...3$\\ \\
	$\bullet$ Empezamos rellenando nuestra tabla: \\
	
		\begin{tabular}{ | c | c | c | c | c | c | c | c| c| }
		
		
		\hline	
		X \ Y & $(0.5,1.5]$ & $(1.5,2.5]$ & $(2.5,5.5]$ & $n_{i.}$& $c_{i.}$& $n_{i.}c_{i.}$ &  $n_{i.}c_{i.}^{2}$ & $x_{i.}\sum_{j=i}^{p}n_{ij}y_{j}$ \\ \hline
		(1,15] & 4 & 2  & 0 & 6 & 8 & 48& 384 & 64 \\
		(15,25] & 1 & 4 & 2 & 7& 20 & 140 & 2800 & 340 \\
		(25,30] & 0 & 3 & 5 & 8 & 27.5 & 220 & 6050 & 715 \\
		$n_{.j}$& 5 & 9  & 7 & 21 & & 408 & 9234 & 1119 \\ 
		$c_{.j}$& 1 & 2  & 4  &  & & & &\\
		$n_{.j}c_{.j}$ & 5 & 18 & 28 & 51 & & & &  \\
		$n_{.j}c_{.j}^{2}$& 5 & 36 & 112 & 153 & & & & \\\hline
		
		
	\end{tabular}
\\ \\

	Calculamos la recta de regresión de Y sobre X:\\
	\\
	$\bullet$  Para ello primero calculamos las medias de X e Y: \\
	
	$\bar{x} = \frac{48 + 140 + 220}{21} = \frac{408}{21} = 19.428 $ °C
	\\ 
	
		$\bar{y} = \frac{5 + 18 + 28}{21} = \frac{51}{21} = 2.428  $ ml de Hg
		\\
	\\
	
	$\bullet$ Tenemos que calcular también las varianzas y la covarianza: \\
	
	
	
		 $v_{x}^{2} =  $$\frac{1}{n}\sum^{k}_{i=1}n_{i.}x_{i}^{2} - \bar{x}^{2} = \frac{9234}{21} - 19.4286^{2} = 62.2438 $ $ °C^{2} $ \\
	 
	 $v_{y}^{2} = $ $\frac{1}{n}\sum^{p}_{j=1}n_{.j}y_{j}^{2} - \bar{y}^{2} = \frac{153}{21} - 2.428^{2} = 1.3876 $ $(ml de Hg)^{2}$ \\
	 
	  $v_{xy} = $ $\frac{1}{n}\sum^{k}_{i=1}\sum^{p}_{j=1}n_{ij}x_{i}y_{j} - \bar{xy}= \frac{1}{21}.1119 - 47.1843 = 6.1014 $ \\
	
	  
	  $\bullet$Ahora que tenemos todos los datos necesarios podemos ya calcular los coeficientes de la recta de regresión :\\ \\
	  $y = ax +b $\\ \\
	  $a = \frac{v_{xy}}{v_{x}^{2}} = \frac{6.1014}{62.2438} = 0,098$\\ \\
	  $b = \bar{y} - a.\bar{x} = 2,4286 - 0,098 .19,4286 = 0,5241$\\ \\
	  $y = 0,098x + 0,5241 $\\ 
	  
	  $\bullet$ Por ultimo calculamos el coeficiente de correlación lineal:\\ \\
	  
	  $r^{2} = \frac{v_{xy}^{2}}{v_{x}^{2}v_{y}^{2}} = \frac{7,1667^{2}}{72,5568 . 1,583} = 0,431055$\\ \\
	  El resultado que hemos obtenido nos indica que la recta de regresión de Y sobre X nos da una información de menos
	  del 45 \% de la variabilidad de Y, luego no seria buena idea suponer una relación lineal entre las variables ya que la bondad de la función es baja. 
	  Pero como podemos ver, el coeficiente de correlación lineal $r = \sqrt{r^{2}} = 0, 6565$ nos indica que hay un significante grado de correlación lineal directa entre las variables.
\end{document}