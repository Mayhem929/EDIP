\problem
Estudiar la dependencia o independencia de las variables en cada una de las siguientes distribuciones. Dar, en cada caso, las curvas de regresión y la covarianza de las dos variables.

    \begin{center}
    \begin{tabular}{| c | c | c | c | c | c | c |}
        \hline
        $X / Y$ & 1 & 2 & 3 & 4 & 5 & $n_i.$\\ \hline
        10 & 2 & 4 & 6 & 10 & 8 & 30 \\
        20 & 1 & 2 & 3 & 5 & 4 & 15 \\
        30 & 3 & 6 & 9 & 15 & 12 & 45 \\
        40 & 4 & 8 & 12 & 20 & 16 & 60 \\
        $n_.j$ & 10 & 20 & 30 & 50 & 40 & 150 \\       \hline
    \end{tabular} \\ 
    \end{center}
    Y es independiente estadísticamente de X, si $f_j^i \equiv f_{j/i}$, es decir, si $\frac{n_{1j}}{n1.} = \frac{n2j}{n2.} = ... = \frac{n_{kj}}{n_{k.}}$ $ \forall j = 1,2,...,p $. En este caso vemos que se cumple. Por ejemplo, para j=1, tenemos: $\frac{2}{30} = \frac{1}{15} = \frac{3}{45} = \frac{4}{60}$, y pasa lo mismo con para j=2,3,4,5. \\
    Así vemos que Y es independiente de X y en este caso, X también es independiente de Y. Aplicando el mismo criterio: $f_i^j \equiv f_{i/j}$, es decir, $\frac{n_{i1}}{n.1} = ... = \frac{n_{ip}}{n.p}$. Tenemos para i=1: $\frac{2}{10} = \frac{4}{20} = \frac{6}{30} = \frac{10}{50} = \frac{8}{40}$, y pasa igual para i=2,3,4. \\ \\
    Como las variables X e Y son independientes, no tiene sentido estudiar la curva de regresión, y sabemos que la covarianza va a ser 0: $\sigma_{xy} = 0$. \\ \\
    \begin{center}
    \begin{tabular}{| c | c | c | c | c | c | c |}
        \hline
        $X / Y$ & 1 & 2 & 3 & $n_i.$ & $n_{i.}x_i$ & $n_{i.}x_i^2$\\ \hline
        -1 & 0 & 1 & 0 & 1 & -1 & 1 \\
        0 & 1 & 0 & 1 & 2 & 0 & 0 \\
        1 & 0 & 1 & 0 & 1 & 1 & 1 \\
        $n_.j$ & 1 & 2 & 1 & 4 & 0 & 2 \\
        $n_{.j}y_j$ & 1 & 4 & 3 & 8 & & \\
        $n_{.j}y_j^2 - \bar{y}^2$ & 1 & 8 & 9 & 18 & & \\
        \hline
    \end{tabular} \\ 
    \end{center}
    Sabemos que estas variables no pueden ser independientes ya que si hay algún $n_{ij}=0$, no se va a dar la igualdad $\frac{n1j}{n1.} = ... = \frac{n_{kj}}{nk.}$ a no ser que $n_{ij}=0$ $\forall i,j$. Esto nos lleva a una contradicción, pues eso sería una variable que no ha presentado distribución de frecuencias. \\
    Sabemos también que X no depende funcionalmente de Y, pues tenemos que $n_{i2} \not = 0$ para más de un i. Por esto mismo, sabemos que Y no depende funcionalmente de X, pues se da $n_{2j} \neq 0$ para más de un j. \\ \\
    Calculamos la covarianza: $\sigma_{xy}$.\\
    \[\bar{x}=\frac{1}{n}\sum_{i=1}^{k}n_{i.}x_i\] 
    \[\bar{y}=\frac{1}{n}\sum_{j=1}^{p}n_{.j}y_j\]
    $\sigma_x^2 = m_{20}-m_{10}^2 = m_{20} = \frac{2}{4} = 0,5$. \\
    $\sigma_y^2 = m_{02}-m_{01}^2 = \frac{18}{4} -2^2=4,5 - 4 = 0,5$. \\
    \[\sigma_{xy} = m_{11} - m_{10}m_{01} = m_{11} = \frac{1}{n} \sum_{i=1}^{k}\sum_{j=1}^{p} x_iy_jn_{ij} = \frac{1*2*-1 + 1*2*1}{4} = \frac{-2+2}{4} = 0 \]
    Calculamos la curva de regresión de X/Y: x=ay+b \\
    $a= \frac{\sigma_{xy}}{\sigma_{y}^2} = 0$ \\
    $b = \bar{x}-a\bar{y} = \bar{x} = 0$ \\ 
    $x=0$ es la curva de regresión de X/Y. \\ \\
    Calculamos ahora la de Y/X: $x=a'y+b'$ \\
    $a' = \frac{\sigma_{xy}}{\sigma_x^2} = 0$ \\
    $b' = \bar{x}-a'\bar{y} = \bar{y} = 2$ \\
    $y=2$ es la curva de regresión de Y/X. \\ \\
    