\problem

De las estad{\'\i}sticas de "Tiempos de vuelo y consumos de
combustible"  de una compa{\~n}{\'\i}a a{\'e}rea, se han  obtenido  datos relativos  a  24  trayectos distintos
realizados por el avi{\'o}n DC-9. A partir de estos datos se han obtenido las siguientes medidas:
$$
\sum y_i =219\mbox{.}719 \ \ \    \sum y_{i}^{2} =2396\mbox{.}504  \ \ \   \sum
x_i y_i =349\mbox{.}486
$$
$$
\sum x_i =31\mbox{.}470  \ \ \    \sum x_{i}^{2} =51\mbox{.}075   \ \ \    \sum
x_{i}^{2} y_i =633\mbox{.}993
$$
$$
\sum x_{i}^{4} =182\mbox{.}977 \ \ \    \sum x_{i}^{3} =93\mbox{.}6
$$
La variable $Y$ expresa el consumo  total  de  combustible,  en  miles  de
libras, correspondiente a un vuelo de  duraci{\'o}n  $X$ (el tiempo  se
expresa en  horas,  y  se  utilizan  como  unidades  de  orden  inferior
fracciones decimales de la hora).

\begin{enumerate}
	\item Ajustar un modelo del tipo $Y=aX+b$.  ?`Qu{\'e} consumo total se estimar{\'\i}a  para un
	programa de vuelos compuesto de 100 vuelos de media hora,
	200 de una hora y 100 de dos horas? ?`Es fiable esta
	estimaci{\'o}n?
	
	\item Ajustar un modelo del tipo  $Y=a+bX+cX^{2}$.
	?`Qu{\'e} consumo total se estimar{\'\i}a para el mismo
	programa de vuelos del apartado a)?
	\item ?`Cu{\'a}l de los dos modelos se ajusta mejor? Razonar la respuesta. \end{enumerate}

En el desarrollo de este ejercicio se va a considerar que a cada tiempo de trayecto de vuelo del avión va asociado una única cantidad de combustible asociada a dicho trayecto (se despreciarán las ligeras variaciones que se puedan dar tanto en el despegue como en el descenso de la aeronave). Por tanto, se presupone que $n_i = 1$, $\forall i = 1,2,3,...,n$.

\subproblem

Los cálculos que se han de hacer serían: 

$$\overline{x} = \frac{1}{n}\sum_{i=1}^{k}n_{i.}·x_i$$
$$\overline{y} = \frac{1}{n}\sum_{j=1}^{p}n_{.j}·y_j$$
$$m_{20} = \frac{1}{n}\sum_{i=1}^{k}n_{i.}·x_i^2$$
$$m_{02} = \frac{1}{n}\sum_{j=1}^{p}n_{.j}·y_j^2$$
$$m_{11} = \frac{1}{n}\sum_{i=1}^{k}\sum_{j=1}^{p}n_{ij}·x_i·y_j$$
$$ \sigma_x^2 = m_{20} - \overline{x}^2$$
$$ \sigma_y^2 = m_{02} - \overline{y}^2$$
$$ \sigma_{xy} = m_{11} - \overline{x}·\overline{y}$$

Pero como se verifica que $n_i = 1$, $\forall i = 1,2,3,...,n$, bastaría con calcular: 

$$\overline{x} = \frac{1}{n}\sum_{i=1}^{k}·x_i$$
$$\overline{y} = \frac{1}{n}\sum_{j=1}^{p}·y_j$$
$$m_{20} = \frac{1}{n}\sum_{i=1}^{k}·x_i^2$$
$$m_{02} = \frac{1}{n}\sum_{j=1}^{p}·y_j^2$$
$$m_{11} = \frac{1}{n}\sum_{i=1}^{k}x_i·y_i$$

Así llegamos a las siguientes conclusiones: $\overline{x} = 1,31125$ h, $\overline{y} = 9,15496$ miles de libras, $m_{20} = 2,128125$ h$^2$, $m_{02} = 99,8543$ miles de libras$^2$, $m_{11} = 14,5619$, $ \sigma_x^2 = 0,4087$ h$^2$, $ \sigma_y^2 = 16,041$ miles de libras$^2$, $ \sigma_{xy} =2,5575$ años$^2$.

Para la recta $x = ay + b$ que minimice los cuadrados de los errores, hemos de calcular los siguientes valores:

$$a = \frac{\sigma_{xy}}{\sigma_x^2}$$
$$b = \overline{y} - a·\overline{x}$$

Sustituyendo en los datos obtenidos, obtenemos: $a=6,2576$ miles de libras/h y $b=0,9497$ miles de libras. Por tanto, la recta de regresión es:

$$ y(x) = 6,2576x + 0,9497 $$ donde x se expresa en horas e y se expresa en miles de libras de combustible. 

Para determinar si el ajuste realizado es fiable, hemos de calcular $R^2 = \dfrac{\sigma_{xy}^2}{\sigma_x^2\sigma_y^2}$. En este caso, su valor es $R^2 = 0,997689$. Ello representa que la estimación es muy fiable, siendo correcta en más de un $99 \%$ de las situaciones. 

El itinerario se presenta en la siguiente tabla: 

\tiny
	\begin{center}
		\begin{tabular}{|c|c|c|c|}
			\hline
			\multicolumn{1}{|l|}{Tiempo (en h)} & \multicolumn{1}{l|}{Combustible / trayecto ( miles de libras)} & \multicolumn{1}{l|}{n.º trayectos} & \multicolumn{1}{l|}{Combustible (miles de libras)} \\ \hline
			0,5 & 4,0785 & 100 & 407,85 \\ 
			1 & 7,2073 & 200 & 1441,46 \\ 
			2 & 13,4649 & 100 & 1346,49 \\ \hline
			\multicolumn{1}{|l|}{} & \multicolumn{1}{l|}{} &  & 3195,8 \\ \hline
		\end{tabular}
	\end{center}
\normalsize

Se emplea un total de 3195,8 miles de libras de combustible para realizar el itinerario.

\subproblem

Con la filosofía de minimizar el cuadrado de los errores para los parámetros de la curva $y = a +bx+cx^2$, tenemos que para calcularlos se ha de resolver el siguiente sistema de ecuaciones:´

\begin{eqnarray}
\nonumber
a+ m_{10}b + m_{20}c = m_{01} \\
\nonumber
m_{10}a +m_{20}b + m_{30}c = m_{11} \\
\nonumber`
m_{20}a +m_{30}b + m_{40}c = m_{21}
\end{eqnarray}

Teniendo en cuenta $m_{30} = 3,9$, $m_{40}=7,624$, $m_{21} = 26,4164$, además de los resultados ya calculados previamente, obtenemos: 

\begin{eqnarray}
\nonumber
a=0,8086 \\
\nonumber
b=6,53 \\
\nonumber`
c=-0,1006
\end{eqnarray}

La ecuación de la curva quedaría $y=0,8086+6,53x-0,1006x^2$. La estimación que realizaría esta regresión al itinerario sería:
\tiny
	\begin{center}
		\begin{tabular}{|c|c|c|c|}
			\hline
			\multicolumn{1}{|l|}{Tiempo (en h)} & \multicolumn{1}{l|}{Combustible / trayecto (miles de libras)} & \multicolumn{1}{l|}{n.º de trayectos} & \multicolumn{1}{l|}{Combustible (miles de libras)} \\ \hline
			0,5 & 4,04845 & 100 & 404,845 \\ 
			1 & 7,238 & 200 & 1447,6 \\ 
			2 & 13,4662 & 100 & 1346,62 \\ \hline
			\multicolumn{1}{|l|}{} & \multicolumn{1}{l|}{} &  & 3199,065 \\ \hline
		\end{tabular}
	\end{center}
\normalsize


\subproblem

No se proporciona suficiente información en el enunciado para determinar cuál de las dos regresiones se ajusta mejor a los datos experimentales. En principio, la regresión lineal minimiza bien los cuadrados de los errores (ajuste casi perfecto, con dependencia lineal), mientras que de la parábola no se puede determinar su coeficiente de correlación. 