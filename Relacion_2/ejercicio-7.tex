%PREAMBULO
\documentclass{article}
\usepackage{pgfplots}

%CUERPO
\begin{document}
    Distribución A: 
    \begin{center}
    \begin{tabular}{| c | c | c | c | c | c | c |}
        \hline
        $X / Y$ & 10 & 15 & 20 & $n_{i.}$ & $n_{i.}x_i$ & $n_{i.}x_i^2 - \bar{x}^2$ \\ \hline
        1 & 0 & 2 & 0 & 2 & 2 & 4,084 \\
        2 & 1 & 0 & 0 & 1 & 2 & 0,184 \\
        3 & 0 & 0 & 3 & 3 & 9 & 0,978 \\
        4 & 0 & 1 & 0 & 1 & 4 & 2,468 \\
        $n_.j$ & 1 & 3 & 3 & 7 & 17 & 7,714\\  
        $n_{.j}y_j$ & 10 & 45 & 60 & 115 & & \\
        $n_{.j}y_j^2 - \bar{y}^2$ & 41,332 & 6,126 & 38,256 & 85,714 & & \\
        \hline
    \end{tabular} \\ 
    \end{center}
    Y depende funcionalmente de X ya que para cada $x_i$ hay un solo $y_j\neq 0$. Sin embargo, X no depende funcionalmente de Y. \\ \\
    Calculamos la curva de regresión de X/Y: 
    \[\bar{x} = \frac{1}{n}\sum_{i=1}^{k}n_{i.}x_i = \frac{17}{7} = 2,429 \]
    \[\bar{y} = \frac{1}{n}\sum_{j=1}^{p}n_{.j}y_j = \frac{115}{7} = 16,429 \]
    \[\sigma_y^2 = m_{02}-m_{01}^2 = \frac{1}{n}\sum_{j=1}^{p}n_{.j}y_j^2-\bar{y}^2 = \frac{85,714}{7} = 12,245\]
    $\sigma_{xy} = m_{11}-m_{10}m_{01} = \frac{2*1*15+1*2*10+3*20*3+1*15*4}{7} - 2,429*16,429 = 1,523$ \\ \\
    Hallamos la curva de regresión de X/Y: =ay+b. \\
    $a = \frac{\sigma_{xy}}{\sigma_y^2} = \frac{1,523}{12,245} = 0,124$ \\
    $b = \bar{x}-a\bar{y} = 2,429 - 0,124*16,429 = 0,392$ \\
    $x=0,125y + 0,375$ \\ \\ 
    
    Distribución B: 
    \begin{center}
    \begin{tabular}{| c | c | c | c |}
        \hline
        $X / Y$ & 10 & 15 & 20 \\ \hline
        1 & 0 & 2 & 0 \\
        2 & 1 & 0 & 0 \\
        3 & 0 & 0 & 3 \\
        \hline
    \end{tabular} \\ 
    \end{center}
    X depende funcionalmente de Y y viceversa ya que para cada $y_j$ hay un solo $x_i\neq 0$ y para cada $x_i$ un solo $y_j\neq 0$ \\ \\
    
    Distribución C:
    \begin{center}
    \begin{tabular}{| c | c | c | c | c | c | c | c |}
        \hline
        $X / Y$ & 10 & 15 & 20 & 25 & $n_{i.}$ & $n_{i.}x_i$ & $n_{i.}x_i^2-\bar{x}^2$\\ \hline
        1 & 0 & 3 & 0 & 1 & 4 & 4 & 2,039 \\
        2 & 0 & 0 & 1 & 0 & 1 & 2 & 0,082 \\
        3 & 2 & 0 & 0 & 0 & 2 & 6 & 3,308 \\
        $n_{.j}$ & 2 & 3 & 1 & 1 & 7 & 12 & 5,429 \\
        $n_{.j}y_j$ & 20 & 45 & 20 & 25 & 110 & & \\
        \hline
    \end{tabular} \\ 
    \end{center}
    X depende funcionalmente de Y, pues para cada $y_j$ hay un solo $x_i\neq 0$. 
    Calculamos la curva de regresión de Y/X: 
    \[\bar{x} = \frac{1}{n}\sum_{i=1}^{k}n_{i.}x_i = \frac{12}{7} = 1,714 \]
    \[\bar{y} = \frac{1}{n}\sum_{j=1}^{p}n_{.j}y_j = \frac{110}{7} = 15,714 \]
    \[\sigma_x^2 = m_{20}-m_{10}^2 = \frac{1}{n}\sum_{i=1}^{k}n_{i.}x_i^2-\bar{x}^2 = \frac{5,429}{7} = 0,776\]
    $\sigma_{xy} = m_{11}-m_{10}m_{01} = \frac{3*15*1+1*1*25+1*2*20+2*3*10}{7} - 1,714*15,714 = 24,286 - 26,934 = -2,648 $ \\ \\
    Hallamos la curva de regresión de X/Y: =ay+b. \\
    $a = \frac{\sigma_{xy}}{\sigma_x^2} = \frac{-2,648}{0,776c} = -3,412$ \\
    $b = \bar{y}-a\bar{x} = 15.714 +3,412*1,714 = 21,562$ \\
    $y=-3,412x + 21,562$ \\ \\ 
    
\end{document}