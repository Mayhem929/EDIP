\problem

Para cada una de las distribuciones:

\hskip 2cm Distribuci{\'o}n A  \hskip 2cm Distribuci{\'o}n B \hskip 2.2cm Distribuci{\'o}n C
\vskip -0.5cm $$
\left .
\begin{array}{c|ccc}
X \backslash Y & 10 & 15 & 20 \\ \hline
1  &  0 &  2 &  0  \\
2  &  1 &  0 &  0  \\
3  &  0 &  0 &  3  \\
4  &  0 &  1 &  0  \\
\end{array}
\right .
\hskip 1cm
\left .
\begin{array}{c|ccc}
X \backslash Y & 10 & 15 & 20 \\ \hline
1  &  0 &  2 &  0  \\
2  &  1 &  0 &  0  \\
3  &  0 &  0 &  3  \\
\end{array}
\right .
\hskip 1cm
\left .
\begin{array}{c|cccc}
X \backslash Y & 10 & 15 & 20  & 25 \\ \hline
1  &  0 &  3 &  0  & 1 \\
2  &  0 &  0 &  1  & 0 \\
3  &  2 &  0 &  0  & 0 \\
\end{array}
\right .
$$

\begin{enumerate}
	\item ?`Dependen funcionalmente $X$ de $Y$ o $Y$ de $X$?
	\item Calcular las curvas de regresi{\'o}n y comentar los resultados.
	
\end{enumerate}

\begin{center}
    \begin{tabular}{|c|c c c c|c|c|c|c|}
        \hline
        $X/Y$ & 1 & 2 & 3 & 4 & $n_i.$ & $n_i.\cdot x_i $ & $n_i.$ \cdot $x_i^2$ & $x_i\sum_{j=1}^p n_{ij} y_j$\\ \hline
        10    & 1 & 3 & 0 & 0 & 4 & 40 & 400 &  70\\
        12    & 0 & 1 & 4 & 3 & 8 & 96 & 1152 & 312\\
        14    & 2 & 0 & 0 & 2 & 4 & 56 & 784 &  140\\
        16    & 4 & 0 & 0 & 0 & 4 & 64 & 1024 & 64\\ \hline
        $n._j$ & 7 & 4 & 4 & 5 & 20 & 256 & 3360 & 586 \\ \hline
        $n._j\cdot y_j$ & 7 & 8 & 12 & 20 & 47 & & &   \\ \hline
        $n._j\cdot y_j^2$ & 7 & 16 & 36 & 80 & 139 & & &  \\ \hline
    \end{tabular}
\end{center}
   
\\
\subproblem
¿Son estadísticamente independientes $X$ e $Y$?\\

Para que se diera la independencia estadística, se tendría que cumplir que las frecuencias de $X$ condicionadas a $Y$ son las mismas para valor de $x_i$. 
$$ \dfrac{n_{i1}}{n._1}=\dfrac{n_{i2}}{n._2}= \dots = \dfrac{n_{ij}}{n._j} = \dots = \dfrac{n_{ip}}{n._p} \qquad \forall i = 1,2,\dots , k.$$
Pero como en algunos casos $n_{ij} = 0$, no se puede dar la independencia ya que no todos son nulos.\\

\subproblem Calcular y representar las curvas de regresión de $X/Y$ e $Y/X$.\\

La curva de regresión tipo 1 de X sobre Y vendrá dada por los puntos $(x_i, \overline{y_i}), i = 1, \dots k$.

$$\{(10, 1.75), (12, 3.25), (14, 2.5), (16, 1)\}$$\\


La curva de regresión tipo 1 de Y sobre X vendrá dada por los puntos $(y_j, \overline{x_j}), i = 1, \dots k$.

$$\{(1, 14.571), (2, 10.5), (3, 12), (4, 12.8) \}$$\\

\subproblem
Cuantificar el grado en que cada variable es explicada por la otra mediante la correspondiente curva de regresión. \\

Con los datos de la tabla:

$$\overline{x} = \dfrac{256}{20} = 12.8 \qquad \overline{y} = \dfrac{27}{20} = 2.35 $$

$$ \sigma_x^2 = m_{20} - m_{10}^2= 4.16 \qquad \sigma_y^2 = m_{02} - m_{01}^2= 1.4275 $$ \\

Calculemos la varianza de los residuos para $Y/X$:
$$\sigma_{ry}^2 = \sum_{i=1}^{k}\sum_{j=1}^p n_{ij}\cdot(y_j-f(x_i))^2 = \dfrac{13.25}{20} = 0.6625 $$

Para $X/Y$:
$$\sigma_{ry}^2 = \sum_{i=1}^{k}\sum_{j=1}^p n_{ij}\cdot(x_i-f(y_j))^2 = \dfrac{37.51}{20} = 1.875 $$

Podemos ahora calcular el coeficiente de correlación:

$$\eta_{\frac{y}{x}}^2 = 1 - \dfrac{\sigma_{ry}}{\sigma_y^2} = 0.536 \qquad 
\eta_{\frac{x}{y}}^2 = 1 - \dfrac{\sigma_{rx}}{\sigma_x^2} = 0.549 $$

Con lo que podemos ver que la bondad está por debajo del 55\%, por lo que la curva no es demasiado precisa. \\
\\

\subproblem 
¿Están X e Y correladas linealmente? Dar las expresiones de las rectas de regresión.

Calculemos el coeficiente de correlación lineal:

$$r =  \dfrac{\sigma_{xy}}{\sigma_x \sigma_y} = -0.32$$

Podemos observar una baja correlación lineal.\\

Calculamos las rectas:

Y sobre X : 
$$y = ax + b$$

$$a = \dfrac{\sigma_{xy}}{\sigma_x^2} = -0.187 \qquad b = \overline{y} - a\overline{x} = 4.75 $$

Recta de Y sobre X $\equiv y = -0.187x + 4.75$\\

X sobre Y :

$$x = ay + b$$

$$a = \dfrac{\sigma_{xy}}{\sigma_y^2} = -0.55 \qquad b = \overline{x} - a\overline{y} = 14 $$

Recta de X sobre Y $\equiv x = -0.55y + 14$ 
