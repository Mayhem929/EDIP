\problem
	 De una muestra de 24 puestos de  venta en  un  mercado  de
	abastos  se ha recogido informaci\'on sobre el n\'umero de balanzas ($X$) y el n\'umero de dependientes ($Y$). Los resultados aparecen en la siguiente tabla:
	$$
	\begin{array}{c|cccc}
		X \backslash Y & 1  &  2 &  3 & 4\\ \hline
		1  &  1 &  2 &  0 & 0 \\
		2  &  1 &  2 &  3 & 1 \\
		3  &  0 &  1 &  2 & 6 \\
		4  &  0 &  0 &  2 & 3 \\
	\end{array}
	$$
	\begin{enumerate}
		\item Determinar las rectas de regresi{\'o}n.
		\item ?`Es apropiado suponer que existe una relaci{\'o}n  lineal  entre  las
		variables?
		\item  Predecir, a partir de los resultados, el n\'umero de balanzas que puede esperarse en un puesto con seis dependientes.  ?`Es fiable esta predicci{\'o}n?
	\end{enumerate}

	En una población de tamaño n = 24 se ha observado dos variables estadísticas,
	X = número de balanzas e Y = número de dependientes, las cuales han
	presentado k = 4, p = 4 modalidades distintas, con distribución de frecuencia
	conjunta $(x_{i},y_{j})$, $n_{ij}$ $ i = 1,...,3$ $ j=1,...3$
	
	$\bullet$ Empezamos rellenando nuestra tabla: \\
	
	\begin{tabular}{ | c | c | c | c | c | c | c | c| c| }
		
		
		\hline	
		X \ Y & 1 & 2 & 3 & 4  & $n_{i.}$ & $n_{i.}x_{i.}$ &  $n_{i.}x_{i.}^{2}$ & $x_{i.}\sum_{j=i}^{p}n_{ij}y_{j}$ \\ \hline
		1 & 1 & 2  & 0 & 0 & 3 & 3 & 3 & 5 \\
		2 & 1 & 2 & 3 & 1& 7 & 14 & 28 & 36 \\
		3 & 0 & 1 & 2 & 6 & 9 & 27 & 81 & 96 \\
		4& 0 & 0  & 2  & 3 & 5 & 20 & 80  & 72 \\
		$n_{.j}$& 2 & 5  & 7 & 10 & 24 & 64 & 192 & 209 \\ 
		$n_{.j}y_{.j}$ & 2 & 10 & 21 & 40 & 73 & & &  \\
		$n_{.j}y_{.j}^{2}$& 2 & 20 & 63 & 160 & 245 & & & \\\hline
		
		
	\end{tabular}
	\\ \\
	
\subproblem	
	$\bullet$ Para ello primero calculamos las medias de X e Y: \\
	
	$\bar{x} = \frac{3 + 14 + 27 +20}{24} = \frac{64}{24} = 2,6667 $ balanzas
	\\ 
	
	$\bar{y} = \frac{2 + 10+21 +40}{24} = \frac{73}{24} = 3,0417  $ dependientes
	\\
	\\
	
	$\bullet$ Tenemos que calcular también las varianzas y la covarianza: \\
	
	
	
	$\sigma_{x}^{2} =  $$\frac{1}{n}\sum^{k}_{i=1}n_{i.}x_{i}^{2} - \bar{x}^{2} = \frac{192}{24} - 2,6667^{2} = 0,8887 $ $ balanzas^{2} $ \\
	
	$\sigma_{y}^{2} = $ $\frac{1}{n}\sum^{p}_{j=1}n_{.j}y_{j}^{2} - \bar{y}^{2} = \frac{245}{24} - 3,0417 ^{2} = 0,9564 $ $dependientes^{2}$ \\
	
	$\sigma_{xy} = $ $\frac{1}{n}\sum^{k}_{i=1}\sum^{p}_{j=1}n_{ij}x_{i}y_{j} - \bar{xy}= \frac{1}{24}.209 - 8,113 = 0,5970 $ \\
	
	
	$\bullet$Ahora que tenemos todos los datos necesarios podemos ya calcular los coeficientes de las rectas de regresión :\\ \\
	\textbf{Para Y / X:}\\
	$y = ax +b $\\ \\
	$a = \frac{\sigma_{xy}}{\sigma_{x}^{2}} = \frac{0,5970}{0,8887} = 0,6718$\\ \\
	$b = \bar{y} - a.\bar{x} = 3,0417 - \frac{0,5970}{0,8887}. 2,6667 = 1,2502$\\ \\
	La recta de regresión de Y sobre X es
	$y = 0,6718 + 1,2502 $\\ \\
	\textbf{Para X / Y:}\\
	$x = ay + b $\\ \\
	$a = \frac{\sigma_{xy}}{\sigma_{y}^{2}} = \frac{0,5970}{ 0,9564 } = 0,6242$\\ \\
	$b = \bar{x} - a.\bar{y} = 2,6667 -  \frac{0,5970}{ 0,9564 }.3,0417 = 0,768$\\ \\La recta de regresión de X sobre Y es
	$x = 0,6242y +  0,768 $\\ 
\subproblem	
	
	
	$\bullet$ Para ello tenemos que calcular el coeficiente de correlación lineal:\\ \\
	
	$r^{2} = \frac{\sigma_{xy}^{2}}{\sigma_{x}^{2}\sigma_{y}^{2}} = \frac{0,5970^{2}}{0,8887 . 0,9564} = 0,4193$\\ \\
 
		$\bullet$El resultado que hemos obtenido nos indica que la recta de regresión de Y sobre X nos da una información de menos
	del 42 \% de la variabilidad de Y, luego no seria buena idea suponer una relación lineal entre las variables ya que la bondad de la función es relativamente baja. 
	Tampoco, el coeficiente de correlación lineal $r =\sqrt{r^{2}} = 0,6475$ nos indica que existe una buena correlación lineal directa entre las variables. \\ \\
	
\subproblem	
	$x = 0,6242y +  0,786 = 0,6242 . 6 +  0,786 = 4,5312 \approx 5 $ balanzas. Si bien la recta de regresión proporcionaría una estimación de la cantidad de balanzas que cabría esperar para 6 dependiente, el resultado no sería fiable, puesto que la recta de regresión tendría únicamente validez para el intervalo muestral estudiado (en este caso, de 1 a 4 dependientes). Más allá de estos límites, no se sabe si la nube de puntos seguiría comportándose como los datos estudiados. 
