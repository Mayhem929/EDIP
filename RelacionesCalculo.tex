\documentclass[12pt,a4paper]{article}
\usepackage[utf8]{inputenc}
\usepackage[T1]{fontenc}
\usepackage{amsmath}
\usepackage{amsfonts}
\usepackage{amssymb}
\usepackage{graphicx}


\usepackage[a4paper,bindingoffset=0.2in,%
left=1in,right=1in,top=1in,bottom=1in,%
footskip=.25in]{geometry}

\begin{document}
	
\title{Relaciones de Ejercicios Cálculo II}
\author{Adrián Jaén Fuentes}
\date{\today}
\maketitle	
\tableofcontents

\newpage

\section{Relación 1}

\begin{center}
	\textsc{{\Large Ejercicio 1}}\\
\end{center}


Estudiar la derivabilidad de la función $f: A \rightarrow \mathbb{R}$, en cada uno de los siguientes casos:

\begin{enumerate}
	\item[a)] $A = [-1,1] $ y $f(x)=\sqrt{1-x^{2}}$
	\item[b)] $A = \mathbb{R} $ y $f(x)=\sqrt[3]{|x|}$
	\item[c)] $A = \mathbb{R} $ y $f(x)=\dfrac{2x}{1+|x|}$
	\item[d)] $A = \mathbb{R}_0^{+} $ y $f(x)=\sqrt{x^{x}}$ si $x \in \mathbb{R}^{+} $ y $f(0) = 0$\\
\end{enumerate}

\begin{center}
	{\Large Solución}\\
\end{center}

\begin{enumerate}
	\item[a)] Empecemos estudiando la continuidad de $f$. Para que la función sea continua en el dominio el argumento ha de ser mayor o igual a 0: \\
	\begin{center}
		$1-x^{2} \geq 0 \leftrightarrow x^{2} \leq 1 \leftrightarrow x \in [-1,1]$
	\end{center}
	Por tanto $f$ es continua en A. Comprobemos ahora su derivabilidad. Para ello obtengamos su derivada:
	\begin{center}
		$f'(x)=\dfrac{-2x}{2\sqrt{1-x^{2}}}$
	\end{center}
	Podemos ver fácilmente que el dominio de esta función será
	$Domf'(x) = Dom(\sqrt{1-x^{2}}) \backslash \{0\}$
	\begin{center}
		$1-x^{2} > 0 \leftrightarrow x^{2} < 1 \leftrightarrow x \in (-1,1)$
	\end{center}
	Por tanto la función $f(x)$ es derivable en $(-1, 1)$
	
	\item[b)] Veamos la continuidad de $f$. Dividamos el valor absoluto.
	
	\begin{center}
		$f(x) = $ 
		$\left\lbrace
		\begin{array}{rcl}
			\sqrt[3]{-x} \quad  x < 0
		\\
		\\  \sqrt[3]{x} \quad x \geq 0
		\end{array}
		\right.$ \\ 
	\end{center}
	Como la raíz cúbica es continua en todo $\mathbb{R}$ y $f_-(0) = f_+(0) = f(0) = 0$, $f$ es continua en todo $\mathbb{R}$. Veamos ahora si es derivable. Tomemos su derivada:
	
	\begin{center}
		$f'(x) = $ 
		$\left\lbrace
		\begin{array}{rcl}
		   \dfrac{-1}{3\sqrt[3]{x^{2}}} \quad  x < 0
		\\
		\\ \dfrac{1}{3\sqrt[3]{x^{2}}} \quad x \geq 0
		\end{array}
		\right.$
	\end{center}
	Podemos ver que existe la derivada en $\mathbb{R} \backslash \{0\}$, sin embargo, $f'(x)$ no está definida en x = 0 por lo que $f$ no es derivable en su dominio.
	\item[c]
 	\item[d]
 	\item[e] hola xd
 	\item[f]
 	\item[g]
 	\item[h]
\end{enumerate}
	
\end{document}