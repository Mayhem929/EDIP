Consideremos un cuerpo que se mueve entre dos puntos $A$ y $B$, alejados una distancia $d > 0$, con una velocidad $v_{1}$ para ir desde A hasta B y una velocidad $v_{2}$ para ir desde B hasta A. La velocidad de un cuerpo es una magnitud física que se define como la rapidez con la que varía la posición de un cuerpo. Su expresión matemática viene dado por la siguiente expresión: $v = \frac{\Delta s}{\Delta t}$, donde $v$ es la velocidad del cuerpo, $s$ es la distancia recorrida y $t$ es el tiempo transcurrido. 

La velocidad media del cuerpo viene determinado por la expresión $v_{media} = \frac{\Delta s}{\Delta t}$. Para determinar dicha velocidad, tengamos en cuenta que el cuerpo recorre dos veces la distancia $d$, una para ir desde $A$ hasta $B$ y otra para ir desde $B$ hasta $A$. Por tanto, sabemos que $s = 2d$. Por otra parte, sabemos que el tiempo invertido sería $t = t_{1} + t_{2}$, donde $t_{1}$ es el tiempo invertido para ir desde A a B y $t_{2}$ es el tiempo invertido para ir desde B hasta A. Por tanto, deducimos que $v_{media} = \frac{2d}{t_{1} + t_{2}}$. Obedeciendo a las expresiones $t_{1} = \frac{d}{v_{1}}$ y $t_{2} = \frac{d}{v_{2}}$, además de dividir numerador y denominador por la distancia $d > 0$, obtenemos:

\begin{center}
	$v_{media} = \frac{2d}{\frac{d}{v_{1}} + \frac{d}{v_{2}}} 	\Rightarrow  v_{media} = \frac{2}{\frac{1}{v_{1}} + \frac{1}{v_{2}}}$
\end{center}

Por tanto, tenemos que la velocidad media del móvil es la \textbf{media armónica} de las dos velocidades. Tomando $v_{1} = 60 \frac{km}{h}$ y $v_{2} = 70 \frac{km}{h}$, tenemos que la velocidad media del recorrido sería:

\begin{equation}
	v_{media} = 64,61 \frac{km}{h}
\end{equation}