\documentclass[11pt]{book}
\usepackage{amssymb}
\usepackage{amsmath}
\usepackage{array}
\usepackage{fancyhea}
\usepackage{supertab}
\usepackage{graphicx}
\usepackage[spanish]{babel}
%\usepackage{float}
\setlength{\textwidth}{17cm} \setlength{\textheight}{25cm}
\setlength{\oddsidemargin}{-0.5cm}
\setlength{\evensidemargin}{-0.5cm}
\setlength{\topmargin}{-1.25cm}
\pagestyle{empty}
\begin{document}
\centerline{\large \bf Relaci{\'o}n de Problemas 1: Variables estad{\'\i}sticas unidimensionales} \smallskip
\centerline{\large \it Estad{\'\i}stica Descriptiva e Introducci{\'o}n a la Probabilidad}

\centerline{Primer curso del Doble Grado en Ingenier\'ia Inform\'atica y Matem{\'a}ticas}
\hrulefill \vskip 0.5cm
\begin{enumerate}
\item El n{\'u}mero de hijos de las familias de una determinada
barriada de una ciudad es una variable estad{\'\i}stica de la que se
conocen los siguientes datos:

\hskip 3cm $\begin{array}{|c|c|c|c|}
  x_{i} & n_{i} & N_{i} & f_{i}
\\   \hline
  0 & 80  &     & 0\mbox{.}16 \\
  1 & 110 &     &      \\
  2 &     & 320 &      \\
  3 &     &     & 0\mbox{.}18 \\
  4 & 40  &     &      \\
  5 &     &     &      \\
  6 & 20  &     &  \\ \hline
\end{array}$   \hskip 1.5cm $\begin{array}{l}  n_i: \ \mbox{frecuencias absolutas} \\ N_i:  \ \mbox{frecuencias absolutas acumuladas}\\ f_i:  \ \mbox{frecuencias relativas}\end{array}$
\begin{enumerate}
    \item Completar la tabla de frecuencias.
    \item Representar la distribuci{\'o}n mediante un diagrama de
    barras y la curva de distribuci{\'o}n.
    \item Promediar los valores de la variable mediante diferentes
    medidas. Interpretarlas.
%    \item Calcular los valores mediano y modal. Interpretarlos.
\end{enumerate}
\vskip 0.3cm \item La puntuaci{\'o}n obtenida por 50 personas que se presentaron a  una  prueba
   de selecci{\'o}n, sumadas las puntuaciones de los distintos tests, fueron:

\smallskip
\centerline{$174,185,166,176,145,166,191,175,158,156,156,187,162,172,197,181,151,$}
\smallskip
\centerline{$161,183,172,162,147,178,176,141,170,171,158,184,173,169,162,172,181,$}
\smallskip
\centerline{$187,177,164,171,193,183,173,179,188,179,167,178,180,168,148,173.$}
\vskip -1cm
\begin{enumerate}
      \item Agrupar los datos en intervalos de amplitud 5 desde 140 a 200 y dar la tabla de frecuencias.
      \item Representar la distribuci{\'o}n mediante un histograma, poligonal de
            frecuencias y curva de distribuci{\'o}n.
\end{enumerate}
 \vskip 0.3cm  \item La distribuci{\'o}n de la renta familiar  en el a{\~n}o 2003 por comunidades aut{\'o}nomas se recoge en la
siguiente tabla:

$\begin{array}{|r|c|c|c|c|c|c|c|}
  \multicolumn{1}{c|}{I_{i}} & n_{i} & N_{i} & f_{i} & F_{i} & c_{i} & a_{i} & h_{i} \\
  \hline
  (8300,9300] & 2 &  &  &  &  &  &  \\
  ,10200] &  & 5 &  &  &  &  &  \\
   &  &  &  & 10/18 &  & 1100 &  \\
   &  &  & 2/18 &  & 12000 &  &  \\
   & 4 &  &  &  &  &  & 0\mbox{.}005/18 \\
   &  & 18 &  &  &  &  & 0\mbox{.}002/18 \\  \hline
\end{array}$ \hskip 1cm $\begin{array}{l}  n_i: \ \mbox{frecuencias absolutas} \\ N_i:  \ \mbox{frec. absolutas acumuladas}\\ f_i:  \ \mbox{frecuencias relativas}
\\ F_i:  \ \mbox{frec. relativas acumuladas}\\ c_i:  \ \mbox{marcas de clase}\\ a_i:  \ \mbox{amplitudes} \\ h_i:  \ \mbox{densidades de frecuencia}\end{array}$
\begin{enumerate}
    \item Completar la tabla.
    \item Representar la distribuci{\'o}n mediante un histograma,
    poligonal de frecuencias y curva de distribuci{\'o}n.
    \item ?`Cu{\'a}ntas comunidades presentan una renta menor o igual
    que 12700 euros? ?`Y cu{\'a}ntas superior a 11300 euros?
\end{enumerate}
\newpage
%\newpage \item Seg{\'u}n el Ministerio de Educaci{\'o}n y Ciencia, el alumnado
%universitario en el curso 2005-2006 se distribuye por {\'a}reas como
%sigue:
%$$\begin{array}{l|c|}
 % \multicolumn{1}{c|}{\mbox{{\'A}rea}} & \mbox{N$^{\underline{o}}$ de alumnos}  \\ \hline
 % \mbox{Humanidades} & 130962 \\
 % \mbox{CC. Sociales y Jur{\'\i}dicas} & 716987 \\
 % \mbox{CC. Experimentales} & 102249 \\
 % \mbox{CC. de la Salud} & 117504 \\
 % \mbox{T{\'e}cnicas} & 374379
% \end{array}$$
% Representa la distribuci{\'o}n mediante un diagrama de rect{\'a}ngulos,
% diagrama de sectores y pictograma.

\vskip 0.4cm  \item En una determinada empresa se realiza un estudio sobre la calidad de su
     producci{\'o}n. La distribuci{\'o}n siguiente informa sobre el n{\'u}mero de  piezas
     defectuosas encontradas en 100 cajas examinadas con 50 unidades cada una
     de ellas:
$$\setlength{\arraycolsep}{10pt}
    \begin{array}{|c||c|c|c|c|c|c|c|c|c|c|c|} \hline
\mbox{N${}^{\underline{o}}$ piezas defectuosas} & 0 & 1 & 2  & 3
& 4  & 5  & 6  & 7 & 8 & 9 & 10 \\ \hline
\mbox{N${}^{\underline{o}}$ de cajas}           & 6 & 9 & 10 & 11
& 14 & 16 & 16 & 9 & 4 & 3 & 2 \\ \hline
   \end{array}
$$

  \begin{enumerate}
     \item Calcular el n{\'u}mero medio de piezas defectuosas por caja.
     \item ?`Cuantas piezas defectuosas se encuentran m{\'a}s frecuentemente en las
        cajas examinadas?
     \item ?`Cu{\'a}l es el n{\'u}mero mediano de piezas defectuosas por caja?
     \item Calcular los cuartiles de la distribuci{\'o}n. Interpretarlos.
     \item Calcular los deciles de orden 3 y 7. Interpretarlos.
     \item Cuantificar la dispersi{\'o}n de la  distribuci{\'o}n utilizando diferentes
           medidas, interpretando los resultados  y se{\~n}alando las ventajas e
           inconvenientes de cada una.
   \end{enumerate}
 \vskip 0.3cm \item Dadas las siguientes distribuciones:

$$\setlength{\extrarowheight}{4pt}
  \begin{array}{|c|c|c|c|c|c|} \hline
      I_{i}^{(1)} & (0,1] & (1,2] & (2,3] & (3,4] & (4,5] \\ \hline
      n_{i}^{(1)} & 12  & 13  &  11 &  8  &  6  \\ \hline
  \end{array}
$$
$$\setlength{\extrarowheight}{4pt}
  \begin{array}{|c|c|c|c|c|c|} \hline
      I_{i}^{(2)} & (0,1] & (1,3] & (3,6] & (6,10] & (10,12] \\ \hline
      n_{i}^{(2)} &  1   &  6  &  7  &  12  &  2  \\ \hline
  \end{array}
$$
%$$\setlength{\extrarowheight}{4pt}
 % \begin{array}{|c|c|c|c|c|c|} \hline
  %    I_{i}^{(3)} & (0,1] & (1,3] & (3,4] & (4,5] & (5,7] \\ \hline
   %   n_{i}^{(3)} &  10  & 15  &  12 & 10  &  7  \\ \hline
  %\end{array}
%$$





Calcular para cada una de ellas:
  \begin{enumerate}
    \item Medias aritm{\'e}tica, arm{\'o}nica y geom{\'e}trica.
    \item El valor m{\'a}s frecuente.
    \item El valor superado por el 50 \% de las observaciones.
    \item Recorrido,   recorrido   intercuart{\'\i}lico   y   desviaci{\'o}n t{\'\i}pica.
          Interpretarlos. ?`Qu{\'e} distribuci{\'o}n es m{\'a}s homog{\'e}nea?
  \end{enumerate}
\vskip 0.3cm

%\item Una empresa de importaci{\'o}n ha realizado cuatro pagos en
%d{\'o}lares a lo largo del a{\~n}o, siendo el cambio aplicado  a  cada
%    operaci{\'o}n y la correspondiente  facturaci{\'o}n trimestral en euros los siguientes:%
%
%$$
%  \begin{array}{ccc}
%\hline
%\mbox{Operaciones} &  \mbox{Tipo cambio} & \mbox{Millones de euros} \\
%              &     \mbox{(euros por d\'olar)}     &  \mbox{cambiados} \\ \hline
%      A       &      0\mbox{.}83    &        45        \\
%      B       &      0\mbox{.}90     &        55       \\
%      C       &      0\mbox{.}87    &        21        \\
%      D       &      0\mbox{.}85    &        40        \\
%  \end{array}
%$$
%
%    Calcular el tipo de cambio medio  en las cuatro operaciones realizadas.
%
%\newpage  \item Una cadena hotelera tiene cinco hoteles   diferentes en distintas ciudades. Los ingresos netos totales durante un a{\~n}o y el rendimiento unitario
%  del capital invertido  en   cada uno  de ellos son los siguientes:
%$$
%    \begin{array}{ccc}
%   \mbox{Hotel}  &  \mbox{Ingresos (euros)}  & \mbox{Rendimientos} \\ \hline
%               1    &     11000        &          0\mbox{.}55          \\
%               2    &     18000        &          0\mbox{.}45       \\
%               3    &     20000        &          0\mbox{.}40        \\
%               4    &     19200        &          0\mbox{.}48         \\
%               5    &      9000        &          0\mbox{.}60         \\
%    \end{array}
%$$%
%
%    Determinar el rendimiento medio  para  el  total  de  los
%    hoteles de la cadena.

 \item Un m{\'o}vil efect{\'u}a un recorrido de 100 km en dos sentidos. En
uno  va a
      una velocidad constante de $V_1$=60 km/h y en el otro va a una velocidad
      constante de $V_2$=70 km/h. Calcular la velocidad media del recorrido.
\vskip 0.3cm  \item Las acciones de una empresa han producido los  siguientes  rendimientos
      netos anuales:
$$
    \begin{array}{cc}
                  \mbox{A{\~n}o} & \mbox{Rentabilidad} \\ \hline
                        1994  &        12\%     \\
                        1995  &        10\%     \\
                        1996  &         7\%     \\
                        1997  &         6\%     \\
                        1998  &         5\%     \\
     \end{array}
$$
    Obtener el rendimiento neto medio en esos cinco a{\~n}os.
\newpage  \item Un profesor califica a sus alumnos seg{\'u}n el criterio siguiente: 40\%  de
      suspensos, 30\% de aprobados, 15\% notables, 10\% sobresalientes y  5\%  de
      matr{\'\i}culas. Las notas obtenidas son las si\-guien\-tes:
$$
\begin{array}{|c|c|c|c|c|c|c|c|c|c|} \hline
  (0,1] & (1,2] & (2,3] & (3,4] & (4,5] & (5,6] & (6,7] & (7,8] & (8,9] & (9,10] \\ \hline
   34 & 74  & 56  &  81 &  94 &  70 &  41 &  28 &  16 &  4  \\ \hline
\end{array}
$$

    Calcular las notas m{\'a}ximas para obtener cada una de las calificaciones.
\vskip 0.4cm  \item Se ha medido la altura de 110 j{\'o}venes, obteniendo:
$$
\begin{array}{|c|c|c|c|c|c|c|c|c|c|} \hline
\mbox{Altura}     & (1\mbox{.}55,1\mbox{.}60] &
(1\mbox{.}60,1\mbox{.}70] & (1\mbox{.}70,1\mbox{.}80] &
(1\mbox{.}80,1\mbox{.}90] & (1\mbox{.}90,2\mbox{.}00] \\ \hline
\mbox{N${}^{\underline{o}}$ j{\'o}venes} & 18 & 31     &  24       &
20    &  17  \\ \hline
\end{array}
$$

  \begin{enumerate}
     \item Si se consideran bajos el 3\% de los individuos de menor  altura,
           ?`cu{\'a}l es la altura m{\'a}xima que pueden alcanzar?
     \item Si se consideran altos el 18\% de los individuos de mayor  altura,
           ?`cu{\'a}l es su altura m{\'\i}nima?
     \item ?`Qu{\'e} altura es superada s{\'o}lo por 1/4 de los j{\'o}venes?
     \item Calcular el n{\'u}mero de j{\'o}venes cuya altura es superior a 1.75.
     \item Calcular la altura m{\'a}xima de los 11 j{\'o}venes m{\'a}s bajos.
     \item Calcular la altura m{\'\i}nima de los 11 j{\'o}venes m{\'a}s altos.
   \end{enumerate}
%\item Sea $X$ la variable estad{\'\i}stica, agrupada en intervalos, que
%toma valores mayores que 0 y cuya pol{\'\i}gonal de frecuencias viene
%dada en la figura siguiente:
%\begin{center}
%\includegraphics[width=10cm,height=5cm]{Problema23_Tema1.bmp}
%\end{center}

 %  \begin{enumerate}
  %    \item Obtener el histograma de frecuencias y la curva de distribuci{\'o}n.
   %   \item Obtener la media aritm{\'e}tica, geom{\'e}trica y arm{\'o}nica.
    %  \item Obtener el valor m{\'a}s com{\'u}n.
     % \item Obtener el valor central de la distribuci{\'o}n.
      %\item Obtener dos cuartiles que est{\'e}n dentro del mismo intervalo.
      %\item Obtener el porcentaje de observaciones entre 17 y 20.
      %\item Obtener las siguientes medidas: recorrido, desviaciones absolutas
       %     medias respecto a la media  y mediana, recorrido intercuart{\'\i}lico,
        %    desviaci{\'o}n t{\'\i}pica. Interpretar sus valores.
   %\end{enumerate}
\vskip 0.4cm  \item Realizando una prueba para el estudio del  c{\'a}ncer  a  150  personas  se
    obtuvo la siguiente tabla seg{\'u}n la edad de los enfermos:
$$
  \begin{array}{|c|c|c|c|c|c|} \hline
 \mbox{Edad}        & (10,30] & (30,40] & (40,50] & (50,60] & (60,90] \\ \hline
 \mbox{N${}^{\underline{o}}$ enfermos} &   15  &   22  &   48  &   40  &  25   \\ \hline
   \end{array}
$$
%Usando el cambio de variable adecuado para transformar las marcas de clase en los valores $0,\ 3, \
%5,\ 7, \ 11$:
  \begin{enumerate}
    \item Calcular la edad m{\'a}s com{\'u}n de los individuos estudiados.
    \item Calcular la edad m{\'\i}nima y m{\'a}xima del 30\% central de los individuos.
    \item Calcular el recorrido intercuart{\'\i}lico y la desviaci{\'o}n t{\'\i}pica.
    \item Calcular e interpretar los valores de los coeficientes de asimetr{\'\i}a y curtosis.
\end{enumerate}
%&\vskip 0.5cm  \item Se ha efectuado un  examen con dos pruebas, A y B,  a  un  grupo  de  alumnos  en  una  materia
 %     determinada, obteni\'endose:
%$$
%\overline{x}_A = 15\mbox{.}5 ,\ \  \overline{x}_B = 75 ,\ \
%\sigma_A = 2\mbox{.}5 ,\ \  \sigma_B = 30\mbox{.}6
%$$
 %   Los alumnos n{\'u}meros 1 y 2 han obtenido:
%$$
%x_{1,A}= 16\mbox{.}7 ,\ \   x_{1,B}= 77\mbox{.}5 ,\ \   x_{2,A}=
%14 ,\ \ x_{2,B} = 82\mbox{.}4
%$$
 %   Convertir las notas de cada alumno en una nota global y decir  cu{\'a}l  de
  %  los dos supera en puntuaci{\'o}n al otro.
%\item Una factor{\'\i}a opera en tres secciones para las que se han calculado  los
 %     siguientes par{\'a}metros, relativos a las horas de funcionamiento  efectivo
  %    por d{\'\i}a:
%$$
 % \begin{array}{|c|c|c|c|} \hline
%\mbox{Secci{\'o}n} & \mbox{Media} & \mbox{Varianza} &
%\mbox{N${}^{\underline{o}}$ obreros} \\ \hline
 %      A    &  364  &   146    &    66      \\
  %     B    &  350  &   182    &    60      \\
   %    C    &  382  &   214    &    68      \\ \hline
   %\end{array}
%$$

 %   Calcular la media y la varianza de las horas de funcionamiento efectivo
 %   de toda la factor{\'\i}a.

\end{enumerate}
\end{document}
