En una población de tamaño $n = 5$ se ha observado una variable estadística $ X = Rentabilidad $  de la empresa durante los años 1994-1998 que ha presentado $k = 5$ modalidades distintas con la siguiente distribución de frecuencias
\\

	\begin{center}
		\begin{tabular}{| c | c | c | c | }
			\hline
			
			$x_{i}$ & $n_{i}$ & $N_{i}$ & $c_{i}$ \\ \hline
			5 & 1  & 1 & 1.05 \\
			6 & 1 & 2 & 1.06 \\
			7 & 1 & 3 & 1.07 \\
			10 & 1 & 4 & 1.1 \\
			12 & 1 & 5 & 1.2 \\ \hline
		\end{tabular}
		
	\end{center}
Como se trata de una variable con rendimientos acumulativos, calcularemos la media geométrica:
\\
\\
$ G = \sqrt[5]{1.05 \cdot 1.06 \cdot 1.07 \cdot 1.1 \cdot 1.12} = 1.079687 $
\\
\\
De lo que significa que el rendimiento neto medio entre 1994 y 1998 es del $7.969\%$.
