Para hacer todos los cálculos de este ejercicio usaremos la siguiente fórmula:

\begin{gather}\tag{Percentil}
    P_{r} = e_{i-1}+\dfrac{\dfrac{nr}{100}-N_{i-1}}{n_{i}}(e_{i}-e_{i-1})
\end{gather}

Donde $r$ es el percentil que queremos calcular, $n$ el tamaño de la población, $n_{i}$ la frecuencia absoluta, $N_{i}$ la frecuencia absoluta acumulada y $e_{i}$ es el final del intervalo $i$ donde encontramos a la cantidad de gente dentro del r\% de la población.

Como vemos necesitaremos algunos datos, como la frecuencia absoluta acumulada. Los calcularemos previamente. Aquí están esos datos de forma tabulada:

\begin{center}
	\begin{table}[htbp]
		\begin{center}
			\begin{tabular}{|c|c|c|}
				\hline
				$I$ & $n_{i}$ & $N_{i}$\\\hline
				(1.55, 1.60] & 18 & 18 \\ \hline
				(1.60, 1.70] & 31 & 49 \\ \hline
				(1.70, 1.80] & 24 & 73 \\ \hline
				(1.80, 1.90] & 20 & 93 \\ \hline
				(1.90, 2.00] & 17 & 110 \\ \hline
			\end{tabular}
		\end{center}
	\end{table}
\end{center}

\subproblem
Si se consideran bajos el 3\% de los individuos de menor altura, ¿cuál es la altura máxima que pueden alcanzar?

Calculamos $P_{3}$. Para ello primero veamos en qué intervalo se encuentra el 3 por ciento de la población con menor altura. Como $0.03*110=3.3$ sabemos que debemos tomar i como 1. El cálculo quedaría así:
\\
\begin{center}
    \begin{*gather}
        P_{3} = 1.55+\dfrac{\dfrac{110*3}{100}-0}{18}(1.6-1.55)=1.559m
    \end{*gather}
\end{center}

Por tanto, deducimos que los jóvenes que midan 1.559 metros o menos son el $3\%$ más bajo de la población.

\subproblem
Si se consideran altos el $18\%$ de los individuos de mayor altura, ¿cuál es su altura mı́nima?

Nos encontramos en el mismo problema que en el apartado anterior, solo que ahora deberemos calcular $P_{82}$, pues $100-18=82$. Como $0.82*110=90.2$ sabemos que $i=4$ ya que es el intervalo cuya frecuencia absoluta acumulada es inmediatamente superior. Hacemos los calculos:
\\
\begin{center}
    \begin{*gather}
        P_{82} = 1.80+\dfrac{\dfrac{110*82}{100}-73}{20}(1.9-1.8)=1.886m
    \end{*gather}
\end{center}

Finalmente, se consideran altos los jóvenes que miden 1.886 metros o más.

\subproblem 
¿Qué altura es superada sólo por 1/4 de los jóvenes?

En este caso nos pregunta qué altura es superada por el 25\% de la población, o lo que es lo mismo, que altura tiene el 75\% más bajo de la población. Para ello calculamos $Q_{75} = P_{75}$. Como $0.75*110=82.5$, por las razones nombradas anteriormente, sabemos que $i = 4$. Sustituimos en la expresión:
\\
\begin{center}
    \begin{*gather}
        P_{75} = 1.80+\dfrac{\dfrac{110*75}{100}-73}{20}(1.9-1.8)=1.847m
    \end{*gather}
\end{center}

Deducimos que la altura superada únicamente por un cuarto de la población es 1.847 metros.

\subproblem
Calcular el número de jóvenes cuya altura es superior a 1.75.

En este apartado nos piden, a fin de cuentas, que hagamos el proceso contrario al que hemos estado haciendo en los apartados anteriores. Antes nos daban $r$ y ahora nos pedían $P_{r}$, ahora nos dan $P_{r}$ y debemos calcular $\dfrac{nr}{100}$. Para ello usamos la expresión del percentil y despejamos:
\\
\begin{center}
    \begin{*gather}
        P_{r} = 1.75 = 1.70+\dfrac{\dfrac{110*r}{100}-49}{24}(1.80-1.70)\\
        \dfrac{nr}{100} = 61\\
    \end{*gather}
\end{center}

Como vemos, hay 61 jóvenes que miden 1.75 o menos, por tanto $110-61=49$ jóvenes superan la latura de 1.75 metros.

\subproblem
Calcular la altura máxima de los 11 jóvenes más bajos.

Primero, esos 11 jóvenes representan el $\dfrac{11}{110}=0.1$ por ciento de la población, por tanto, debemos calcular $D_{1} = P_{10}$. Sabemos que los 11 jóvenes más bajos están en el primer intervalo, por lo que $i=1$ y sustituimos:

\begin{center}
    \begin{*gather}
        P_{10} = 1.55+\dfrac{\dfrac{110*10}{100}-0}{18}(1.60-1.55)=1.581m
    \end{*gather}
\end{center}

Luego, la altura máxima de los 11 jóvenes más bajos es 1.581 metros.

\subproblem 
Calcular la altura mínima de los 11 jóvenes más altos.

En este apartado seguiremos el mismo procedimiento que en el apartado anterior. Como los 11 jóvenes representan el $0.1$ por ciento de la población, debemos calcular $D_{9} = P_{90}$, pues recordemos que son los 11 más altos. Teniendo en cuenta que $110-11=99$ sabemos que $i=5$. Procedemos a realizar los cálculos:

\begin{center}
    \begin{*gather}
        P_{90} = 1.90+\dfrac{\dfrac{110*90}{100}-93}{17}(2.00-1.90)=1.935m
    \end{*gather}
\end{center}

En conclusión, la altura mínima de los 11 jóvenes más altos es de 1.935 metros. 
