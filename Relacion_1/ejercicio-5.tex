Escribamos las tablas con los datos que nos pueden interesar para la resolución del ejercicio: \\

\begin{tabular}{| c | c | c | c |}
	\hline
	$I_i^{(1)}$ & $c_i^{(1)}$ & $n_i^{(1)}$ & $N_i^{(1)}$ \\ \hline
	$(0, 1]$ & $0.5$ & $12$ & $12$ \\
	$(1, 2]$ & $1.5$ & $13$ & $25$ \\
	$(2, 3]$ & $2.5$ & $11$ & $36$ \\
	$(3, 4]$ & $3.5$ & $8$ & $44$ \\
	$(4, 5]$ & $4.5$ & $6$ & $50$ \\ \hline
	
\end{tabular} \\\\\\

\begin{tabular}{| c | c | c | c |}
	\hline
	$I_i^{(2)}$ & $c_i^{(2)}$ & $n_i^{(2)}$ & $N_i^{(2)}$ \\ \hline
	$(0, 1]$ & $0.5$ & $1$ & $1$ \\
	$(1, 3]$ & $2$ & $6$ & $7$ \\
	$(3, 6]$ & $4.5$ & $7$ & $14$ \\
	$(6, 10]$ & $8$ & $12$ & $26$ \\
	$(10, 12]$ & $11$ & $2$ & $28$ \\ \hline
	
\end{tabular}\\
\\ 
\\
\textbf{Apartado a)}\\

La media aritmética de una variable es la suma de sus valores entre en número total de observaciones. Como los datos están organizados en intervalos de clase, para calcular la media aritmética vamos a suponer que todos los datos de un intervalo son idénticos a la marca de clase de cada intervalo. Por tanto, la media la calculamos de la siguiente manera: 
\begin{center}
	$\overline{x} = \dfrac{1}{n}\sum_{i=1}^{k}n_i·c_i$
\end{center}

Para la primera distribución, $n = 50$, luego la media aritmética es 2.082. 

Para la segunda, $n = 28$, luego la media aritmética será 5.786. \\


La media armónica se usa para promediar datos de magnitudes relativas. La definimos como la inversa de la media aritmética de los valores inversos de la variable (en nuestro caso, usamos las marcas de clase):
\begin{center}
	$H = \dfrac{n}{\dfrac{n_1}{c_1}+\dfrac{n_2}{c_2}+ ...+\dfrac{n_k}{c_k}} = \dfrac{n}{\sum_{i=1}^{k}\dfrac{n_i}{c_i}}$
\end{center}

Para la primera distribución: H = 1.223.

Para la segunda distribución: H = 3.399.  \\


Finalmente la media geométrica se usa cuando se desea promediar datos de una variable que tiene efectos
multiplicativos acumulativos en la evolución de una determinada característica con un
valor inicial fijo.

Es la raíz n-ésima del producto de los $n$ valores (o marcas de clase) de la distribución:

\begin{center}
	$G = \sqrt[n]{\prod_{i=1}^{k}c_i^{n_i}}$
\end{center}

Para no perder precisión en el resultado, la calcularemos sabiendo que el logaritmo de la media geométrica es la media aritmética de los logaritmos de los valores de la variable:

\begin{center}
	$\log G = \log \sqrt[n]{\prod_{i=1}^{k}c_i^{n_i}} = \dfrac{1}{n}\sum_{i=1}^{k}n_i \log c_i$
\end{center}

Para la primera distribución: G = 1.685.

Para la segunda distribución: G = 1.562.  \\


\textbf{Apartado b)}\\

Se nos pide calcular la moda de las distribuciones, es decir, el valor  de mayor frecuencia: \\

En el caso primero, observamos que se encuentra en el intervalo (1,2], luego haremos un promedio teniendo en cuenta los intervalos contiguos. Para ello apliquemos el algoritmo suponiendo que es una distribución continua.

\begin{center}
	$M_O = e_{i-1} + \dfrac{h_i - h_{i-1}}{2h_i - h_{i-1} - h_{i+1}} (e_i - e_{i-1})$
\end{center}

Por tanto, la moda será: $M_O = 1.\overline{3}$.\\


En el segundo caso la moda proviene del intervalo (6, 10] luego apliquemos el mismo método teniendo en cuenta que $e_{i-1} = 6, e_i=10, h_{i-1} = 7, h_i = 12, h_{i+1} = 2 $ \\ 
$M_O = 7.\overline{3}$\\
\\

\textbf{Apartado c)}\\

Para calcular la mediana hemos de observar las frecuencias absolutas acumuladas; tendremos que encontrar el punto que divida a la población en dos partes iguales, luego buscamos el punto bajo el que se encuentre la mitad de la población, es decir, $n/2$:\\

Para la primera distribución: $n/2 = 25$ luego el $50\%$ de la población supera el valor 2.

Para la segunda distribución: $n/2 = 14$ luego el $50\%$ de la población supera el valor 6.\\
\\

\textbf{Apartado d)}\\

\begin{enumerate}
	\item Recorrido: Es la amplitud del intervalo en la que se mueven los valores de la variable, por tanto, se calcula restando el valor mas grande posible menos el más pequeño. Como en nuestro caso tenemos intervalos, cojamos los valores de las cotas: \\
	Para la primera distribución será 5 y para la segunda será 12.
	\item Recorrido intercuartílico: Para calcularlo necesitamos los cuartiles 1 y 3:
	
	\begin{center}
		$Qk = e_{i-1} + \dfrac{\dfrac{n}{k} - N_{i-1}}{n_i}(e_i-e_{i-1})$
	\end{center}
	
	Para la distribución 1 será $Q_1 = 1.038, Q_3 = 3.187; R_I = 2.1$
	
	Para la distribución 2 será $Q_1 = 3, Q_3 = 7.75; R_I = 4.75$\\
	
	\item Desviación típica: es una medida de dispersión, una medida de cómo los valores individuales de el conjunto pueden diferir de la media. Es la media de los valores absolutos de las distancias de cada uno de los valores de la variable a la media de la distribución. El valor absoluto se usa para evitar que las desviaciones de signo contrario se cancelan mutuamente. También se puede calcular como la raíz cuadrada positiva de la varianza.
	\begin{center}
		$\sigma^2 = \dfrac{1}{n} \sum_{i=1}^{n}(x_i-\overline{x})^2$
	\end{center}
	
	Para la distribución 1 será $\sigma^2 = 1.75; \sigma = 1.323$
	
	Para la distribución 1 será $\sigma^2 = 4.64; \sigma = 2.154$\\
	
	
	Para ver cuál es mas homogénea, calculemos el Coeficiente de variación de Pearson:
	
	$C.V._1 = \dfrac{1.323}{2.082}= 0.635$ \\
	\\
	\\
	$C.V._2 = \dfrac{2.154}{5.786}= 0.372$
	
	La segunda distribución es más homogénea.
	
\end{enumerate}