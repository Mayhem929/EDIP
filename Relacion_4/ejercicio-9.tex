\problem
\ item  $ N $ urnas contienen cada una 4 bolas blancas y 6 negras, mientras otra
urna contiene 5 blancas y 5 negras. De las $ N + 1 $ urnas se elige una al
azar y se extraen dos bolas sucesivamente, sin reemplazamiento,
resultando ser ambas negras. Sabiendo que la probabilidad de que
queden 5 blancas y 3 negras en la urna elegida es 1/7, encontrar $ N $ .
	
	\begin{flushleft}
		A = "Elegir la urna N+1" \\
		B = "Extraer 2 bolas negras sucesivamente sin remplazamiento" \\
		Luego tenemos que calcular la probabilidad de A condicionada a B:\\ \\
	\end{flushleft}
	
\begin{flushleft}
	
		$P(A|B) = \frac{P(B|A)P(A)}{P(B|A)P(A)+P(B|\bar{A})P(\bar{A})} $\\ \\
\end{flushleft}
Remplazamos las probabilidades con los datos que tenemos y obtenemos que: \\ \\
$P(A|B) = \frac{0,5 \cdot \frac{4}{5} \cdot \frac{1}{N+1}}{N\cdot0,6\cdot\frac{5}{9}\frac{1}{N+1} + 0,5\cdot\frac{4}{9}\cdot\frac{1}{N+1}} $\\\\
Despejando N obtenemos que $N=4$
