\documentclass{article}

\begin{document}
Ejercicio 1. En una batalla naval, tres destructores localizan y disparan simultáneamente a un submarino. La probabilidad de que el primer destructor acierte el disparo es 0'6, la de que lo acierte el segundo es 0'3 y la de que lo acierte el tercero es 0'1. ¿Cuál es la probabilidad de que el submarino sea alcanzado por algún disparo? \\ \\
Ri: acierta el disparo el destructor i.\\
P(R1) = 0'6, P(R2) = 0'3, P(R3) = 0'1 \\
$P(R1 \cup R2 \cup R3) = P(R1) + P(R2) + P(R3) - P(R1 \cap R2) - P(R1 \cap R3) - P(R2 \cap R3) + P(R1 \cap R2 \cap R3) = 0'6+0'3+0'1-(0'6*0'1)-(0'6*0'3)-(0'1*0'3)+(0'6*0'3*0'1) = 0'748 $\\
\end{document}