\problem
Se lanza una moneda; si sale cara, se introducen  $k$ bolas blancas en una urna
y si sale cruz, se introducen  $2k$ bolas blancas. Se hace una  segunda  tirada,
poniendo en la urna $h$ bolas negras si sale cara y $2h$ si sale cruz. De la urna  as{\'\i}
compuesta se toma una bola al azar. ?`Cu{\'a}l es la probabilidad de que sea
negra?

\textbf{Aclaraciones a añadir al enunciado:} La urna se considera en un principio vacía y se supone que la moneda está trucada. 

Sean $N$ el suceso de sacar una bola negra de la urna resultante del ejercicio y $A_{x}$ el suceso de obtener tras el lanzamiento de la moneda cara si  $x=C$ o cruz si $x=X$. Dado que la moneda está trucada, podemos suponer que $P(A_C) = p$ y $P(A_X) = 1 - p$. Se nos pide determinar $P(N)$. Para ello, aplicamos el teorema de la probabilidad total: 


\begin{equation} \label{ej-12}
\begin{split}
P(N) & =  P(N|A_C \cap A_C) · P(A_C \cap A_C) + P(N|A_C \cap A_X) · P(A_C \cap A_X) \\
& + P(N|A_X \cap A_C) · P(A_X \cap A_C) + P(N|A_X \cap A_X) · P(A_X \cap A_X)
\end{split}
\end{equation}

Para hallar cada probabilidad condicionada, aplicamos la regla de Laplace (casos favorables entre casos posibles):

\begin{equation*}
\begin{split}
P(N|A_C \cap A_C) & = \frac{h}{k+h} \\
P(N|A_C \cap A_X) & = \frac{2h}{k+2h} \\
P(N|A_X \cap A_C) & = \frac{h}{2k+h} \\
P(N|A_X \cap A_X) & = \frac{2h}{2k+2h} = \frac{h}{k+h}
\end{split}
\end{equation*}

Como cada lanzamiento de moneda es independiente, deducimos que $P(A_x \cap A_y) = P(A_x) · P(A_y)$. Basta ahora sustituir en la expresión \ref{ej-12} y operar, dando como resultado:

$$P(N) = p^2·\frac{h}{k+h} + p·(1-p)·(\dfrac{2h}{k+2h} + \frac{h}{2k+h}) + (1-p)^2·\frac{h}{k+h}$$