\problem

\ item   Un estudiante debe pasar durante el curso 5 pruebas selectivas. La
probabilidad de pasar la primera es 1/6. La probabilidad de pasar la
$ i $ - { \ ' e} sima, habiendo pasado las anteriores es   $ 1 / ( 7 -i) $ . Determinar la
probabilidad de que el alumno apruebe el curso.

	\begin{flushleft}
		De los datos del enunciado sabemos que la probabilidad de aprobar la prueba i-esima es: $P(A_{i}) = \frac{1}{7-i}$
	\end{flushleft}

\begin{flushleft}
	Los sucesos son independientes ya que el hecho de aprobar una prueba o no aprobarla no influye en la probabilidad de aprobar el resto de las pruebas, entonces la probabilidad de aprobar el curso es: 
\end{flushleft}

$P(\cap^{5}_{i=1} A_{i}) = P(A_{1})\cdot P(A_{2})\cdot P(A_{3})\cdot P(A_{4})\cdot P(A_{5}) = $ \\ $=\frac{1}{6} \cdot \frac{1}{5} \cdot \frac{1}{4} \cdot \frac{1}{3} \cdot \frac{1}{2} = \frac{1}{720} = 0,001389$

