\problem
Ejercicio 8. La probabilidad de que se olvide inyectar el suero a un enfermo durante la ausencia del doctor es 2/3. Si se le inyecta el suero, el enfermo tiene igual probabilidad de mejorar que de empeorar, pero si no se le inyecta, la probabilidad de mejorar se reduce a 0'25. Al regreso, el doctor encuentra que el enfermo ha empeorado. ¿Cuál es la probabilidad de que no se le haya inyectado el suero? \\ \\
M: enfermo ha mejorado \\
S: se le ha inyectado suero \\
Utilizamos la regla de Bayes: $P(A|B) = \frac{P(B|A)*P(A)}{\sum P(B|A)*P(A)}$ \\ \\
$P(\overline{S}) = \frac{2}{3}$, $P(M|S) = 0'5$, $P(\overline{M}|S) = 0'5$, $P(M|\overline{S}) = 0'25$ \\ 
$P(\overline{S}|\overline{M}) = \frac{P(\overline{S})*P(\overline{M}|\overline{S})}{P(\overline{M})} = \frac{\frac{2}{3}*0'75}{0'5*\frac{1}{3}+0'75*\frac{2}{3}}= \frac{3}{4} = 0'75$