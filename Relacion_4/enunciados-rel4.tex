\documentclass[11pt]{book}
\usepackage{amssymb}
\usepackage{amsmath}
\usepackage{array}
%\usepackage{fancyhea}
%\usepackage{supertab}
\usepackage{graphicx}
\usepackage[spanish]{babel}
\usepackage{float}
\setlength{\textwidth}{17cm} \setlength{\textheight}{25cm}
\setlength{\oddsidemargin}{-0.5cm} \setlength{\evensidemargin}{-0.5cm}
\setlength{\topmargin}{-1.25cm} \pagestyle{empty}
\begin{document}
\centerline{\large \bf Relaci{\'o}n de Problemas 4: Probabilidad condicionada e independencia
de sucesos}
\smallskip \centerline{\large \it Estad{\'\i}stica Descriptiva e Introducci{\'o}n a la
Probabilidad}

\centerline{Primer curso del Doble Grado en Ingenier\'ia Inform\'atica y Matem{\'a}ticas}
\hrulefill \vskip 0.5cm

\begin{enumerate}



\item  En una batalla naval, tres destructores localizan y disparan simult{\'a}neamente a  un
    submarino. La proba\-bi\-li\-dad de que el primer  destructor  acierte el disparo es 0.6,  la de que lo acierte el segundo es 0.3 y  la  de  que  lo
    acierte el tercero es 0.1. ?`Cu{\'a}l es la probabilidad  de  que  el  submarino  sea alcanzado por alg{\'u}n disparo?

\item  Un estudiante debe  pasar durante el curso  5  pruebas  selectivas.  La
    probabilidad de pasar la primera es 1/6. La probabilidad  de  pasar  la
    $i$-{\'e}sima, habiendo pasado  las  anteriores  es  $1/(7-i)$.  Determinar  la
    probabilidad de que el alumno apruebe el curso.

\item  En una ciudad, el 40\% de las personas tienen pelo rubio, el 25\%  tienen
    ojos azules y el 5\% el pelo rubio y los ojos azules. Se selecciona  una
    persona al azar. Calcular la probabilidad de los siguientes sucesos:
    \begin{enumerate}
      \item tener el pelo rubio si se tiene los ojos azules,
      \item tener los ojos azules si se tiene el pelo rubio,
      \item no tener pelo rubio ni ojos azules,
      \item tener exactamente una de estas caracter{\'\i}sticas.
    \end{enumerate}

\item  En una poblaci{\'o}n de moscas, el 25\% presentan mutaci{\'o}n  en  los
    ojos, el 50\% presentan mutaci{\'o}n en las  alas,  y  el  40\%  de  las  que
    presentan mutaci{\'o}n en los ojos presentan mutaci{\'o}n en las alas.
    \begin{enumerate}
      \item ?`Cu{\'a}l es la probabilidad de que una mosca elegida al  azar  presente
            al menos una de las mutaciones?
      \item ?`Cu{\'a}l es la probabilidad de que   presente  mutaci{\'o}n
            en los ojos pero no en las alas?
    \end{enumerate}

\item Una empresa utiliza dos sistemas alternativos, $A$ y $B$, en  la  fabricaci{\'o}n
    de un art{\'\i}culo, fabricando  por el sistema $A$    el  20\%  de  su  producci{\'o}n.
    Cuando a un cliente se le ofrece dicho art{\'\i}culo, la probabilidad de que
    lo compre es 2/3 si {\'e}ste se fabric{\'o} por el sistema $A$  y   2/5  si  se
    fabric{\'o} por  el  sistema  $B$.  Calcular  la  probabilidad  de  vender  el
    producto.



\item Se consideran dos urnas: la primera con 20 bolas, de las cuales  18
    son blancas, y  la segunda con 10 bolas, de las cuales  9  son  blancas.  Se
    extrae una bola de la segunda urna y  se  deposita  en  la  primera; si   a
    continuaci{\'o}n, se extrae una bola de {\'e}sta,  calcular  la  probabilidad  de
    que sea blanca.


\item Se dispone de tres  urnas con  la siguiente  composici{\'o}n de bolas blancas y negras:

    \hskip 2cm $U_{1}$: 5B y 5N $\ \ \ $ $U_{2}$: 6B y 4N  $\ \ \ $
    $U_{3}$: 7B y 3N.

    Se elige  una urna al azar y se sacan  cuatro  bolas sin reemplazamiento.
    \begin{enumerate}
      \item Calcular la probabilidad de que las cuatro sean blancas.
      \item Si en las  bolas extra{\'\i}das s{\'o}lo hay una negra, ?`cu{\'a}l es la probabilidad de que la urna elegida haya sido $U_{2}$?
    \end{enumerate}

\item  La probabilidad de que se olvide inyectar el suero a un enfermo  durante
    la ausencia del doctor es 2/3. Si se le inyecta el suero, el  enfermo
    tiene igual probabilidad de mejorar que de empeorar, pero si no se le inyecta, la probabilidad de mejorar se reduce a  0.25.
    Al regreso, el  doctor
     encuentra que el enfermo ha empeorado. ?`Cu{\'a}l es la  probabilidad
    de que no se le haya inyectado el  suero?

\item $N$ urnas contienen cada una 4 bolas blancas y 6  negras,  mientras  otra
    urna contiene 5 blancas y 5 negras. De las $N+1$ urnas se  elige  una  al
    azar y se  extraen  dos  bolas  sucesivamente, sin  reemplazamiento,
    resultando ser ambas  negras. Sabiendo  que  la  probabilidad  de  que
    queden 5 blancas y 3 negras en la urna elegida es 1/7, encontrar $N$.


\item Se dispone de  6 cajas,  cada una con 12 tornillos;
    una caja tiene 8 buenos y 4 defectuosos; dos  cajas  tienen  6  buenos  y  6
    defectuosos y tres cajas tienen 4 buenos y 8 defectuosos. Se  elige  al
    azar una caja y se extraen 3  tornillos  con  reemplazamiento,  de  los
    cuales 2 son buenos y 1 es defectuoso. ?`Cu{\'a}l es la probabilidad de  que
    la caja elegida contuviera 6 buenos y 6 defectuosos?

\item  Se seleccionan $n$ dados con probabilidad $p_{n} =1/2^{n}, \ n\in \Bbb N$.
    Si se lanzan estos $n$
    dados y se obtiene una suma de 4 puntos, ?`cu{\'a}l es la probabilidad de  haber  seleccionado
 4 dados?

\item  Se lanza una moneda; si sale cara, se introducen  $k$ bolas blancas en una urna
    y si sale cruz, se introducen  $2k$ bolas blancas. Se hace una  segunda  tirada,
    poniendo en la urna $h$ bolas negras si sale cara y $2h$ si sale cruz. De la urna  as{\'\i}
    compuesta se toma una bola al azar. ?`Cu{\'a}l es la probabilidad de que sea
    negra?


\end{enumerate}
\end{document}
