\problem

Se dispone de tres  urnas con  la siguiente  composici{\'o}n de bolas blancas y negras:

\hskip 2cm $U_{1}$: 5B y 5N $\ \ \ $ $U_{2}$: 6B y 4N  $\ \ \ $
$U_{3}$: 7B y 3N.

Se elige  una urna al azar y se sacan  cuatro  bolas sin reemplazamiento.
\begin{enumerate}
	\item Calcular la probabilidad de que las cuatro sean blancas.
	\item Si en las  bolas extra{\'\i}das s{\'o}lo hay una negra, ?`cu{\'a}l es la probabilidad de que la urna elegida haya sido $U_{2}$?
\end{enumerate}

\subproblem
A = Sacar blanca en la primera urna

B = Sacar blanca en la segunda urna

C = Sacar blanca en la tercera urna \\


$P(Sacar\ 4 \ blancas) = P(U_1)P(A|U_1) + P(U_2)P(B|U_2) + P(U_3)P(C|U_3) =  \\
\dfrac{1}{3}(\dfrac{5}{10}\cdot\dfrac{4}{9}\cdot\dfrac{3}{8}\cdot\dfrac{2}{7}) + 
  \dfrac{1}{3}(\dfrac{6}{10}\cdot\dfrac{5}{9}\cdot\dfrac{4}{8}\cdot\dfrac{3}{7}) + 
  \dfrac{1}{3}(\dfrac{7}{10}\cdot\dfrac{6}{9}\cdot\dfrac{5}{8}\cdot\dfrac{4}{7}) = 0.0873$ \\

\subproblem

$A_i$ = Coger una urna i 

$B$ = Sacar 4 y que solo una sea negra \\

$P(A_i) = \dfrac{1}{3}$

$P(B|A_1) = \dfrac{5}{10} \cdot \dfrac{4}{9} \cdot \dfrac{3}{8} \cdot \dfrac{5}{7} = \dfrac{5}{84} = 0.0595$ \\

$P(B|A_2) = \dfrac{6}{10} \cdot \dfrac{5}{9} \cdot \dfrac{4}{8} \cdot \dfrac{4}{7} = \dfrac{2}{21} = 0.0952$ \\

$P(B|A_3) = \dfrac{7}{10} \cdot \dfrac{6}{9} \cdot \dfrac{5}{8} \cdot \dfrac{3}{7} = \dfrac{1}{8} = 0.125$ \\


Para calcular $P(A_2|B)$ usamos la regla de Bayes:

$$ P(A_2|B) = \dfrac{P(B|A_2)P(A_2)}{P(B|A_1)P(A_1)+P(B|A_2)P(A_2)+P(B|A_3)P(A_2)} = \dfrac{\frac{2}{21 \cdot 3}}{\frac{1}{3}(\frac{5}{84} + \frac{2}{21} + \frac{1}{8})} = \dfrac{16}{47}$$
