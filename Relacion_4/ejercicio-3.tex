\documentclass{article}

\begin{document}
Ejercicio 3. En una ciudad, el 40\% de las personas tienen el pelo rubio, el 25\% tienen ojos azules y el 5\% el pelo rubio y los ojos azules. Se selecciona una persona al azar. Calcular la probabilidad de los siguientes sucesos: \\ \\
R: tener el pelo rubio \\
A: tener los ojos azules \\
Utilizamos la fórmula de la probabilidad condicionada para el apartado a y b: $P(A|B) = \frac{P(A\cap B}{P(B)}$. \\ \\
a) tener el pelo rubio si se tiene los ojos azules \\
$P(R|A) = \frac{P(R \cap A)}{P(A)} = \frac{0'05}{0'25} = 0'2$ \\ \\
b) tener los ojos azules si se tiene el pelo rubio \\
$P(A|R) = \frac{P(R \cap A)}{P(R)} = \frac{0'05}{0'4} = 0'125$ \\ \\
c) no tener pelo rubio ni ojos azules \\ 
$P(A \cup R) = 1 - (P(A)+P(R)-P(A\cap R)) = 1 - (0'4+0'25-0'05) = 0'4 $ \\ \\
d) tener exactamente una de estas características \\
$P((A\cap \overline{B}) \cup (\overline{A} \cap B)) = 0'4 - 0'05 + 0'25 - 0'05 = 0'55 $ \\ 
\end{document}