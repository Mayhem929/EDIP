\problem
\ item Una empresa utiliza dos sistemas alternativos, $ A $ y $ B $ , en la fabricaci { \ ' o} n
de un art { \ ' \ i } culo, fabricando por el sistema $ A $     el 20 \%   de su producci { \' o} n.
Cuando a un cliente se le ofrece dicho art { \ ' \ i } culo, la probabilidad de que
lo compre es 2/3 si { \ ' e} ste se fabric { \' o} por el sistema $ A $   y 2/5 si se
fabric { \ ' o} por el sistema   $ B $ . Calcular la probabilidad de vender el
producto.
{\tiny }
		De los datos del enunciado sabemos que la probabilidad de usar el sistema de producción A es  $20\%$ luego la probabilidad de usar  el sistema de producción B es  $80\%$   
		\begin{flushleft}
			A = "Usar el sistema de producción A " \\
			B = "Usar el sistema de producción B" \\
			C = "Comprar un producto fabricado por el sistema de producción A" \\
			D = "Comprar un producto fabricado por el sistema de producción B"\\  \\
		
		\end{flushleft}
		Entonces la probabilidad de vender un producto es:\\ \\
	$P = P(A) \cdot P(C|A) + P(B) \cdot P(D|B) = 0,2 \cdot\frac{2}{3} + 0,8 \cdot\frac{2}{5} = \frac{34}{75} = 0,4533 $

