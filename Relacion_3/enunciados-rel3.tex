\documentclass[11pt]{book}
\usepackage{amssymb}
\usepackage{amsmath}
\usepackage{array}
%\usepackage{fancyhea}
%\usepackage{supertab}
\usepackage{graphicx}
\usepackage[spanish]{babel}
%\usepackage{float}
\setlength{\textwidth}{17cm} \setlength{\textheight}{25cm}
\setlength{\oddsidemargin}{-0.5cm} \setlength{\evensidemargin}{-0.5cm}
\setlength{\topmargin}{-1.25cm} \pagestyle{empty}
\begin{document}
\centerline{\large \bf Relaci{\'o}n de Problemas 3: Espacios de probabilidad}
\smallskip \centerline{\large \it Estad{\'\i}stica Descriptiva e Introducci{\'o}n a la
Probabilidad}

\centerline{Primer curso del Doble Grado en Ingenier\'ia Inform\'atica y
Matem{\'a}ticas} \hrulefill \vskip 0.5cm
\begin{enumerate}

\item Durante un a{\~n}o,  las  personas  de  una  ciudad  utilizan  3  tipos  de
    transportes: metro (M),  autob{\'u}s  (A),  y  coche  particular  (C).  Las
    probabilidades  de  que  durante  el  a{\~n}o  hayan  usado  unos  u  otros
    transportes son:

    M:\ 0.3; \ \ A:\ 0.2; \ \ C: \ 0.15; \ \ M y A: \ 0.1; \ \ M y C:\ 0.05; \ \ A y C:\ 0.06; \ \ M,\ A y C: \ 0.01

    Calcular las probabilidades siguientes:
    \begin{enumerate}

      \item que una persona viaje en metro y no en autob{\'u}s;
       \item que una persona tome al menos dos medios de transporte;
       \item que una persona viaje en metro o en coche, pero no en autob{\'u}s;
      \item que viaje en metro,  o bien en autob{\'u}s y en coche;
      \item que una persona vaya a pie.
    \end{enumerate}

\item Sean $A, B$ y $C$ tres sucesos de un espacio probabil{\'\i}stico ($\Omega , \cal{A} ,\mbox{P}$),  tales
    que  $P(A)=0.4, \  P(B)=0.2, \  P(C)=0.3, \  P(A\cap B)=0.1$  y $(A\cup B)\cap C= \emptyset$.
    Calcular las probabilidades de los siguientes sucesos:
    \begin{enumerate}
      \item s{\'o}lo ocurre $A$,
      \item ocurren los tres sucesos,
      \item ocurren $A$ y $B$ pero no $C$,
      \item por lo menos dos ocurren,
      \item ocurren dos y no m{\'a}s,
      \item no ocurren m{\'a}s de dos,
      \item ocurre por lo menos uno,
      \item ocurre s{\'o}lo uno,
      \item no ocurre ninguno.
    \end{enumerate}

\item Se sacan dos  bolas  sucesivamente  sin  devoluci{\'o}n  de  una  urna  que
    contiene 3 bolas rojas distinguibles y 2 blancas distinguibles.
    \begin{enumerate}
      \item Describir el espacio de probabilidad asociado a este experimento.
      \item Descomponer en sucesos elementales los sucesos: {\em la primera bola  es
       roja}, {\em la segunda bola es blanca} y calcular la probabilidad de cada uno de ellos.
      \item
            ?`Cu{\'a}l es la probabilidad de que ocurra alguno de los sucesos considerados en el  apartado anterior?
    \end{enumerate}


\item  Una urna contiene $a$ bolas blancas y $b$ bolas  negras.  ?`Cu{\'a}l  es  la
    probabilidad de que  al  extraer  dos  bolas  simult{\'a}neamente  sean  de
    distinto color?

\item  Una urna contiene 5 bolas  blancas  y  3  rojas.  Se  extraen  2  bolas
    simult{\'a}neamente. Calcular la probabilidad de obtener:
    \begin{enumerate}
      \item dos bolas rojas,
      \item dos bolas blancas,
      \item una blanca y otra roja.
    \end{enumerate}

\item  En una loter{\'\i}a de 100 billetes hay 2 que tienen premio.
    \begin{enumerate}
      \item ?`Cu{\'a}l es la probabilidad de ganar al menos un premio si  se  compran
            12 billetes?
      \item ?`Cuantos billetes habr{\'a} que comprar  para  que  la  probabilidad  de
            ganar al menos un  premio sea mayor que 4/5?
    \end{enumerate}

\item  Se consideran los 100 primeros n{\'u}meros naturales. Se sacan 3 al azar.
    \begin{enumerate}
      \item Calcular la probabilidad de que en los 3 n{\'u}meros obtenidos no exista
            ning{\'u}n cuadrado perfecto.
      \item Calcular la probabilidad de que exista al menos un cuadrado perfecto.
      \item Calcular la probabilidad de que exista un  s{\'o}lo  cuadrado  perfecto,
            de que existan  dos, y la de que los tres lo sean.
    \end{enumerate}


\item  En una carrera de relevos cada equipo  se  compone  de  4  atletas.  La
    sociedad deportiva  de  un  colegio  cuenta  con  10  corredores  y  su
    entrenador debe formar un equipo de relevos que disputar{\'a} el campeonato,
    y el orden en que participar{\'a}n los seleccionados.
    \begin{enumerate}
      \item ?`Entre cu{\'a}ntos equipos distintos habr{\'a} de elegir el entrenador si los
       10 corredores son de igual val{\'\i}a? (Dos equipos  con los mismos atletas en
       orden distinto se consideran diferentes)
      \item Calcular  la  probabilidad  de  que   un   alumno   cualquiera   sea
       seleccionado.
    \end{enumerate}

\item Una tienda compra bombillas en lotes de 300 unidades. Cuando  un  lote  llega,  se
    comprueban 60 unidades elegidas al azar, rechaz{\'a}ndose el  env{\'\i}o  si  se
    supera la cifra de 5 defectuosas. ?`Cu{\'a}l es la probabilidad  de  aceptar
    un lote en el que haya 10 defectuosas?

\item Una secretaria debe echar al correo $3$ cartas; para ello, introduce cada carta en
      un sobre y escribe las direcciones al azar. ?`Cu{\'a}l es la probabilidad de que al
      menos una carta llegue a su destino?
\end{enumerate}
\end{document}
