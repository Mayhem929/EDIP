\problem
En una carrera de relevos cada equipo  se  compone  de  4  atletas.  La
sociedad deportiva  de  un  colegio  cuenta  con  10  corredores  y  su
entrenador debe formar un equipo de relevos que disputar{\'a} el campeonato,
y el orden en que participar{\'a}n los seleccionados.
\begin{enumerate}
	\item ?`Entre cu{\'a}ntos equipos distintos habr{\'a} de elegir el entrenador si los
	10 corredores son de igual val{\'\i}a? (Dos equipos  con los mismos atletas en
	orden distinto se consideran diferentes)
	\item Calcular  la  probabilidad  de  que   un   alumno   cualquiera   sea
	seleccionado.
\end{enumerate}
\subproblem
Siguiendo el esquema de combinatoria llegamos a que : \\ \\
$ \star $No intervienen todos los corredores ya que intervienen solo 4 atletas

$ \star $ Importa el orden 

$ \star $ No se pueden repetir personas   \\

Entonces nos encontramos ante una variación sin repeticiones, entonces: \\ \\
$V^{p}_{n} = \frac{n!}{(n-p)!}$\\ 
$V^{4}_{10} = \frac{10!}{(10-4)!} = 7\cdot8\cdot9\cdot10 = 5040$ equipos posibles\\ \\
	
\subproblem
La probabilidad de que un alumno sea seleccionado es: \\ \\
$P = \frac{4}{10} = 0,4$
