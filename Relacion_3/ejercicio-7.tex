\problem
A := 1º es cuadrado perfecto

B := 2º es cuadrado perfecto 

C := 3º es cuadrado perfecto 

D := Ninguno es perfecto \\


\subproblem 
Calcular la probabilidad de que en los 3 números obtenidos no exista ningún cuadrado perfecto. \\

% subconjuntos de 3 elementos de entre los 100 posibles; combinaciones sin repetición

Buscamos $P(D) = P(\overline{A\cup B \cup C}) = P(\overline{A}\cap \overline{B} \cap \overline{C}) =  P(\overline{A})\cdot P(\overline{B}) \cdot P(\overline{C})$

Hay 10 cuadrados perfectos entre los 100 primeros números naturales, luego

$$P(D) = \dfrac{10}{100}\cdot \dfrac{9}{99}\cdot\dfrac{8}{98} = \dfrac{117480}{161700} = 0.7265$$


Luego la probabilidad de que entre 3 números que escojamos entre los 100 primeros naturales no haya ningún cuadrado perfecto es de: $$ P(D) = P(\overline{A\cup B \cup C}) = \dfrac{117480}{161700} = 0.7265$$

\subproblem 
Calcular la probabilidad de que exista al menos un cuadrado perfecto.\\
Se nos pide $P(\overline D)$, que es simplemente 
$$1 - P(D) = 0.2735$$

\subproblem
Calcular la probabilidad de que exista un sólo cuadrado perfecto, de que existan dos, y la de que los tres lo sean.
\begin{itemize}
    \item P(1 perfecto) = \\
    P(1º perfecto $\cap$ 2º no perfecto $\cap$ 3º no perfecto) + \\
    P(1º no perfecto $\cap$ 2º perfecto $\cap$ 3º no perfecto) + \\ 
    P(1º no perfecto $\cap$ 2º no perfecto $\cap$ 3º perfecto); \\
    $P(A\cap \overline{B} \cap \overline{C}) + 
     P(\overline{A}\cap B \cap \overline{C}) + 
     P(\overline{A}\cap \overline{B} \cap C)  = \\
     P(A)\cdot P(\overline{B}) \cdot P(\overline{C}) + 
     P(\overline{A})\cdot P(B) \cdot P(\overline{C}) +
     P(\overline{A})\cdot P(\overline{B}) \cdot P(C) = \\
    \dfrac{10}{100}\cdot \dfrac{90}{99} \cdot \dfrac{89}{98} + 
    \dfrac{90}{100}\cdot \dfrac{10}{99} \cdot \dfrac{89}{98} + 
    \dfrac{90}{100}\cdot \dfrac{89}{99} \cdot \dfrac{10}{98}= 0.2477$ \\
    
    \item P(3 perfectos) = P(1º perfecto $\cap$ 2º perfecto $\cap$ 3º perfecto) = \\ 
    $P(A) \cdot P(B) \cdot P(C) =  \dfrac{10}{100} \cdot\dfrac{9}{99} 
    \cdot\dfrac{8}{98} = \dfrac{2}{2695} = 0.0007421$ \\
    
    \item P(2 perfectos); Se podría hacer de forma similar que para 1 perfecto pero como sabemos que los conjuntos son disjuntos (no se pueden dar a la vez dos a dos) y que la probabilidad de que se haya al menos un cuadrado es $P(\overline D) = \\ 
    P(A\cup B \cup C) = 0.2735$, entonces tenemos que $P(\overline D) = P(A) + P(B) + P(C) \\ 
    P(B) = P(\overline D) - P(A) - P(C) = 0.02505$
\end{itemize}
