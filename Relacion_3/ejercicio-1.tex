\problem

Durante un a{\~n}o,  las  personas  de  una  ciudad  utilizan  3  tipos  de
transportes: metro (M),  autob{\'u}s  (A),  y  coche  particular  (C).  Las
probabilidades  de  que  durante  el  a{\~n}o  hayan  usado  unos  u  otros
transportes son:

M:\ 0.3; \ \ A:\ 0.2; \ \ C: \ 0.15; \ \ M y A: \ 0.1; \ \ M y C:\ 0.05; \ \ A y C:\ 0.06; \ \ M,\ A y C: \ 0.01

Calcular las probabilidades siguientes:
\begin{enumerate}
	\item que una persona viaje en metro y no en autob{\'u}s;
	\item que una persona tome al menos dos medios de transporte;
	\item que una persona viaje en metro o en coche, pero no en autob{\'u}s;
	\item que viaje en metro,  o bien en autob{\'u}s y en coche;
	\item que una persona vaya a pie.
\end{enumerate}

\subproblem
En este apartado debemos calcular la probabilidad de la intersección entre M y $\overline{A}$:

\begin{equation*}
    P(M \cap \overline{A}) = P(M-A) = P(M) - P(M \cap A) = 0.3 - 0.1 = 0.2
\end{equation*}

\subproblem
Aquí debemos considerar todas las formas posibles de tomar dos medios de transporte. Calculamos:

\begin{gather*}
    P((A \cap M) \cup (A \cap C) \cup (M \cup C) \cup (A \cup M \cup C)) =\\ P(A \cap M) + P(A \cap C) + P(M \cup C) - 2P(A \cup M \cup C) = 0.2 + 0.05 + 0.06 - 2 \cdot 0.01 = 0.19
\end{gather*}

\subproblem
Ahora debemos calcular la probabilidad de la unión entre las intersecciones de $\overline{A}$ con M y C:

\begin{gather*}
    P((M \cap \overline{A}) \cup (C \cap \overline{A})) = P(M \cap \overline{A}) + P(C \cap \overline{A}) - P(C \cap M \cap \overline{A}) = \\P(M \cap \overline{A}) + P(C) - P(C \cap A) - P(M \cap C) + P(C \cap M \cap A) = 0.2 + 0.15 - 0.06 - 0.05 + 0.01 = 0.25
\end{gather*}

\subproblem
En este apartado debemos calcular la unión entre M y la intersección de A y C:

\begin{equation*}
    P(M \cup (A \cap C)) = P(M) + P(A \cap C) - P(M \cap A \cap C) = 0.3 + 0.06 - 0.01 = 0.35
\end{equation*}

\subproblem
En este último apartado, para calcular lo que nos piden debemos considerar que no hay otro transporte además de los dados en el ejercicio. Si hubiera otro medio, como barco o patinete, no tendríamos los datos suficientes para hacer los cálculos. Hecha esta suposición, calculamos la probabilidad que nos piden:

\begin{gather*}
    P(\overline{A} \cap \overline{M} \cap \overline{C}) = \overline{P(A \cup M \cup C)} = 1 - P(A \cup M \cup C) = \\ P(A) + P(M) + P(C) - P(A \cap M) - P(A \cap C) - P(M \cap C) + P(A \cap M \cap C) =\\ 1 - (0.2 + 0.3 + 0.15 - 0.1 - 0.06 - 0.05 + 0.01) = 0.55
\end{gather*}
