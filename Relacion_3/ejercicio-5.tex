\documentclass{article}

\begin{document}
Ejercicio 5. Una urna contiene 5 bolas blancas y 3 rojas. Se extraen 2 bolas simultáneamente. Calcular la probabilidad de obtener: \\ \\
a) dos bolas rojas \\
A1 = primera roja, A2 = segunda roja \\
$P(A1 \cap A2) = \frac{3}{8} * \frac{2}{7} = 0'375*0'286 = 0'107$ \\ \\
b) dos bolas blancas \\ 
B1 = primera blanca, B2 = segunda roja \\
$P(B1 \cap B2) =  \frac{5}{8} * \frac{4}{7} = 0'625 * 0'571 = 0'357$ \\ \\
c) una blanca y otra roja \\
$P((A1 \cap B2)\cup (A2 \cap B1)) = \frac{3}{8}*\frac{5}{7} + \frac{5}{8}*\frac{3}{7} = 0'536$ \\
\end{document}