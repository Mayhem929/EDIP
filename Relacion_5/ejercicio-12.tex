\problem

Sea $X$ una variable aleatoria sim{\'e}trica con respecto al
punto 2, y con coeficiente de variaci{\'o}n 1. ?`Qu{\'e} puede decirse acerca de las
siguientes probabilidades?:
\begin{itemize}
\item $P(-8<X<12)$
\item $P(-6<X<10).$
\end{itemize}

Antes de empezar, tomaremos los datos que nos dan para deducir la desviación típica y la esperanza matemática. Para empezar nos dicen que el coeficiente de variación es 1, lo que implica que:
\begin{equation*}
    C.V_X = 1 \Rightarrow \dfrac{\sigma_X}{E[X]} = 1 \Rightarrow \sigma_X = E[X]
\end{equation*}

Además, nos dicen que la variable aleatoria es simétrica con respecto al punto 2. Esto implica que el punto que deja a la misma cantidad de valores a su derecha que a su izquierda es dos, por lo que $E[X] = 2$ y además:
\begin{equation*}
    E[X] = 2 \Rightarrow \sigma_X = E[X] = 2
\end{equation*}

Ahora, para abordar el ejercicio, usaremos la desigualdad de Chebychev de la siguiente forma:
\begin{equation*}
    P(|X-E[X]| \geq k \sigma_X) = P(E[X] - k\sigma_X < k < E[X] + k\sigma_X) \geq 1 - \dfrac{1}{k^2}
\end{equation*}

\subproblem

Vamos a calcular $k$ para adaptar la desigualdad de Chebychev a esta probabilidad:
\begin{equation*}
    E[X] + 2k = 12 \Rightarrow k = \dfrac{12-2}{2} = 5
\end{equation*}

Finalmente sustituimos $k=5$ en la expresión anterior:
\begin{equation*}
    P(E[X] - k\sigma_X < k < E[X] + k\sigma_X) = P(-8 < x < 12) \geq 1 - \dfrac{1}{5^2} = 0.96
\end{equation*}

Luego, con los datos que tenemos, podemos decir que $P(-8 < x < 12) \geq 0.96$.

\subproblem

Vamos a calcular $k$ para adaptar la desigualdad de Chebychev a esta probabilidad:
\begin{equation*}
    E[X] + 2k = 10 \Rightarrow k = \dfrac{10-2}{2} = 4
\end{equation*}

Finalmente sustituimos $k=4$ en la expresión anterior:
\begin{equation*}
    P(E[X] - k\sigma_X < k < E[X] + k\sigma_X) = P(-6 < x < 10) \geq 1 - \dfrac{1}{4^2} = 0.9375
\end{equation*}

Luego, con los datos que tenemos, podemos decir que $P(-6 < x < 10) \geq 0.9375$.
