\problem
\ vskip 0.75cm \ item Fonctions calculées   de la densité de las variables $ Y = 2
X + 3 $  \ y \  $ Z = | X | $ , siendo   $ X $ una variable continua con funci { \ ' o} n de
densité
$$
f_X (x) =   \ frac {1} {4}, \ \ - 2 <x < 2 .
$$
\subproblem\\ \\	
\begin{flushleft}
	
		Sea X, una variable aleatoria continua, con f(x) su función de densidad y E=(-2,2).
\end{flushleft}
\\
	Como Y=h(X), entonces $h^{-1}(y) = X = \frac{Y-3}{2} $\\ \\
	Calculamos la imagen de h(x) en los extremos del intervalo E=(-2,2):\\
	$h(-2)= -1 $\\
	$h(2)= 7 $
	\\ \\
Por lo tanto $E^{'} = (-1, 7)$\\ \\
Sea:\\
 $g(y) = f(h^{-1}(y)) . |(h^{-1})^{'}(y)|$, si $y\in (-1,7)$\\
$g(y) = 0 $ en otro caso.\\ \\
Calculamos g(y):\\
$g(y) = f(\frac{Y-3}{2}) . |\frac{1}{2}| = \frac{1}{4}.\frac{1}{2} = \frac{1}{8}$ ,  si $y\in (-1,7)$ \\ \\
Veamos ahora la variable Z. Como Z=h(X), entonces $h^{-1}(z) = X $\\ \\
Z=$|X|$=\left\lbrace\begin{array}{c} -x~si~x~<=~0\\  x~si~x~>~0 \end{array}\right.\\

\subproblem\\ \\
\begin{flushleft}
	Calculamos la imagen de h(E) en los extremos del intervalo E=(0,2):
\end{flushleft}
\\ \\
$g(y) = f(h^{-1}(y)) . |(h^{-1})^{'}(y)| + f(h^{-1}(y)) . |(h^{-1})^{'}(y)| = \frac{1}{4}.1 + \frac{1}{4}.1 = \frac{1}{2} $, si\\ $z\in(0,2)$\\
\\ \\
 g(y)=\left\lbrace\begin{array}{c} \frac{1}{2}~~si~0~<~x~<~2\\ 0~~~en~otro~caso \end{array}\right.\\


