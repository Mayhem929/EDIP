\problem \\
\subproblem \\
Tenemos que la variable X puede tomar 3 valores: 0, 1, 2 (No sale ninguna bola blanca, sale una o salen dos). Por tanto: \\

$P[X = 0] = \dfrac{2}{10}\cdot\dfrac{1}{9} = \dfrac{1}{45} \quad
P[X = 1] = \dfrac{8}{10}\cdot\dfrac{2}{9} = \dfrac{16}{45} \quad 
P[X = 2] = \dfrac{8}{10}\cdot\dfrac{7}{9} = \dfrac{28}{45}$ \\

Ahora podemos definir la función de distribución: \\

$$
F(X) = \left\{
     \begin{array}{lr}
       0 & \mbox{si }  x < 0 \\
       \\
       \dfrac{1}{45} & \mbox{si }0 <= x < 1 \\
       \\
       \dfrac{17}{45}& \mbox{si }  1 <= x < 2 \\
       \\
       1 & \mbox{si } x >= 1 
     \end{array}
   \right.
$$
\\

\subproblem \\
\begin{itemize}
\item Media: Es la esperanza matemática, es decir, el centro de gravedad de la distribución de probabilidad y se calcula como $$E[X] = \sum_i x_i \cdot P[X = x_i] = \dfrac{16}{45} + 2\cdot \dfrac{28}{45} = \dfrac{8}{5}$$

\item Mediana: Es el valor de la variable que deja por debajo la el 50\% de la probabilidad. En nuestro caso, mirando a la función de distribución tenemos que $$ \mbox{Me} = 2 $$

\item Moda: Es el valor de $x_i$ con mayor probabilidad, por lo que se calculará como el máximo de los $P[X = x_i]$. Por tanto $$ \mbox{Mo} = 2$$
\end{itemize} \\

\subproblem \\
Para el recorrido intercuartílico calculemos los percentiles 75 y 25: \\

$$ P_{75} = \mbox{min}\{\ x_i\ /\ P[X < x_i] = F[x_i] > 0.75\} = x_2 = 2 $$
$$ P_{25} = \mbox{min}\{\ x_i\ /\ P[X < x_i] = F[x_i] > 0.25\} = x_1 = 1 $$

Por tanto el recorrido intercuartílico será $Q_3 - Q_1 = 2-1 = 1$, que indica la longitud del intervalo en el que se encuentra el 50\% de los datos.
