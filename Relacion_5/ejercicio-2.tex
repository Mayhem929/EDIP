\problem

Sea $X$ el n{\'u}mero de bolas blancas obtenidas al sacar dos de una urna con 10 bolas de las que 8 son blancas.
Calcular: \begin{itemize} \item [a)] Funci{\'o}n masa de probabilidad y  funci{\'o}n de
	distribuci{\'o}n. \item [b)] Media, mediana y moda, dando la interpretaci{\'o}n de cada
	una de estas medidas.\item [c)]Intervalo intercuart{\'\i}lico,  especificando  su
	interpretaci{\'o}n.\end{itemize}

\subproblem
Tenemos que la variable X puede tomar 3 valores: 0, 1, 2 (No sale ninguna bola blanca, sale una o salen dos). Por tanto: \\

$P[X = 0] = \dfrac{2}{10}\cdot\dfrac{1}{9} = \dfrac{1}{45} \quad
P[X = 1] = \dfrac{8}{10}\cdot\dfrac{2}{9} = \dfrac{16}{45} \quad 
P[X = 2] = \dfrac{8}{10}\cdot\dfrac{7}{9} = \dfrac{28}{45}$ \\

Ahora podemos definir la función de distribución: \\

$$
F(X) = \left\{
     \begin{array}{lr}
       0 & \mbox{si }  x < 0 \\
       \\
       \dfrac{1}{45} & \mbox{si }0 \leq x < 1 \\
       \\
       \dfrac{17}{45}& \mbox{si }  1 \leq x < 2 \\
       \\
       1 & \mbox{si } x \geq 2
     \end{array}
   \right.
$$
\\

\subproblem
\begin{itemize}
\item Media: Es la esperanza matemática, es decir, el centro de gravedad de la distribución de probabilidad y se calcula como $$E[X] = \sum_i x_i \cdot P[X = x_i] = \dfrac{16}{45} + 2\cdot \dfrac{28}{45} = \dfrac{8}{5} \approx 2$$

De esta medida obtenemos que es esperable sacar 2 bolas blancas de la urna. 

\item Mediana: Es el valor de la variable que deja por debajo al 50\% de la probabilidad. En nuestro caso, mirando a la función de distribución tenemos que $$ \mbox{Me} = 2 $$

Como no hay ninguna probabilidad que deje debajo exactamente al 50\% de las probabilidades, únicamente se ha podido adoptar el valor 2, que es el máximo valor que puede tomar la variable estadística. 

De esta medida podemos interpretar que la distribución de probabilidad no es uniforme, concentrando casi todas las probabilidades a la derecha $X=2$. 

\item Moda: Es el valor de $x_i$ con mayor probabilidad, por lo que se calculará como el máximo de los $P[X = x_i]$. Por tanto $$ \mbox{Mo} = 2$$

De esta medida podemos concluir que, después de realizar el experimento aleatorio, lo más común es que ambas bolas extraídas sean blancas. 
\end{itemize}

\subproblem
Para el recorrido intercuartílico calculemos los percentiles 75 y 25: \\

$$ Q_3 = P_{75} = \mbox{min}\{\ x_i\ /\ P[X < x_i] = F[x_i] > 0.75\} = x_2 = 2 $$
$$ Q_1 = P_{25} = \mbox{min}\{\ x_i\ /\ P[X < x_i] = F[x_i] > 0.25\} = x_1 = 1 $$

Por tanto el recorrido intercuartílico será $Q_3 - Q_1 = 2-1 = 1$, que indica la longitud del intervalo en el que se encuentra el 50\% de los datos.
