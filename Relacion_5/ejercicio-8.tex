\problem

Calcular las funciones masa de probabilidad de las
variables $Y = X + 2$ y $Z = X^{2}$, siendo $X$ una variable aleatoria  con
distribuci{\'o}n:
$$
P(X=-2)= \displaystyle\frac{1}{5}, \hskip 0.5cm P(X=-1)=
\displaystyle\frac{1}{10}, \hskip 0.5cm P(X=0)= \displaystyle\frac{1}{5},
\hskip 0.5cm P(X=1)= \displaystyle\frac{2}{5}, \hskip 0.5cm P(X=2)=
\displaystyle\frac{1}{10}\cdot
$$
?`C{\'o}mo afecta el cambio de $X$ a $Y$ en el coeficiente de variaci{\'o}n?

En primer lugar, notemos que la variable aleatoria X es discreta. Basta aplicar el teorema de cambio de variable:

\begin{equation*}
P(y_i) = \left\{ \begin{array}{lcc}
1/5   & y_i = 0 \\
1/10  & y_i = 1 \\
1/5   & y_i = 2 \\
2/5   & y_i = 3 \\
1/10  & y_i = 4 \\
\end{array}
\right.
\end{equation*}

\begin{equation*}
P(z_i) = \left\{ \begin{array}{lcc}
1/5   & z_i = 0 \\
1/2   & z_i = 1 \\
3/10  & z_i = 4 \\
\end{array}
\right.
\end{equation*}

Notemos que:

\begin{equation*}
\begin{split}
EY & = \sum_i P(Y=y_i)·y_i = \sum_i P(Y=y_i)·(x_i - a) = \sum_i P(X=x_i)·(x_i - a) \\
& = \sum_i P(X=x_i)·x_i - a\sum_i P(X=x_i) = \sum_i P(X=x_i)·x_i - a \\
& = EX - a
\end{split}
\end{equation*}

Por tanto, se verifica que: 

\begin{equation*}
\begin{split}
E[(Y-EY)^2] & = \sum_i P(Y=y_i)(y_i - EY)^2 = \sum_i P(X=x_i)(x_i - a - EX + a)^2 \\
& = \sum_i P(X=x_i)(x_i - EX)^2 = E[(X-EX)^2]
\end{split}
\end{equation*}

Deducimos que el coeficiente de variación no ha de variar. 

Por otra parte, sabemos que $C.V.(X) = \frac{E[(X-EX)^2]}{EX}$ y $C.V.(Y) = \frac{E[(Y-EY)^2]}{EY}$. Dividiendo ambas expresiones, tenemos que $C.V.(X) = \frac{EY}{EX}·C.V.(Y)$. Operamos:

\begin{center}
	\begin{tabular}{|r|r|r|r|r|}
		\hline
		\multicolumn{1}{|c|}{} & \multicolumn{1}{c|}{$p(x_i)$} & \multicolumn{1}{c|}{$p(x_i)·x_i$} & \multicolumn{1}{c|}{$(x_i-EX)^2$} & \multicolumn{1}{c|}{$p(x_i)(x_i-EX)^2$} \\ \hline
		-2 & 0,2 & -0,4 & 4,41 & 0,882 \\ \hline
		-1 & 0,1 & -0,1 & 1,21 & 0,121 \\ \hline
		0 & 0,2 & 0 & 0,01 & 0,002 \\ \hline
		1 & 0,4 & 0,4 & 0,81 & 0,324 \\ \hline
		2 & 0,1 & 0,2 & 3,61 & 0,361 \\ \hline
		\multicolumn{1}{|l|}{} & \multicolumn{1}{l|}{} & EX = 0,1 & \multicolumn{1}{l|}{} & $E[(X-EX)^2]$=1,69 \\ \hline
	\end{tabular}
\end{center}

\begin{center}
	\begin{tabular}{|r|r|r|r|r|}
		\hline
		\multicolumn{1}{|c|}{} & \multicolumn{1}{c|}{$p(y_i)$} & \multicolumn{1}{c|}{$p(y_i)·y_i$} & \multicolumn{1}{c|}{$(y_i-EY)^2$} & \multicolumn{1}{c|}{$p(y_i)(y_i-EY)^2$} \\ \hline
		0 & 0,2 & 0 & 4,41 & 0,882 \\ \hline
		1 & 0,1 & 0,1 & 1,21 & 0,121 \\ \hline
		2 & 0,2 & 0,4 & 0,01 & 0,002 \\ \hline
		3 & 0,4 & 1,2 & 0,81 & 0,324 \\ \hline
		4 & 0,1 & 0,4 & 3,61 & 0,361 \\ \hline
		\multicolumn{1}{|l|}{} & \multicolumn{1}{l|}{} & $EY = 2,1$ & \multicolumn{1}{l|}{} & $E[(Y-EY)^2]$ = 1,69 \\ \hline
	\end{tabular}
\end{center}

Deducimos que $C.V.(X) = 21·C.V.(Y)$. 
