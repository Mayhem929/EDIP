\problem

La dimensi{\'o}n en cent{\'\i}metros  de  los tornillos  que salen
de cierta f{\'a}brica es una  variable aleatoria, $X$, con funci{\'o}n de densidad
     $$
     f(x) = \displaystyle{\frac{k}{x^{2}}},\ \    1 \leq x \leq 10.
     $$
     \begin{enumerate}
       \item [$a)$]Determinar el valor de $k$, y obtener la funci{\'o}n de distribuci{\'o}n.
       \item [$b)$] Hallar la probabilidad de que la dimensi{\'o}n de un tornillo
             est{\'e}  entre 2 y 5 cm.
       \item [$c)$] Determinar la  dimensi{\'o}n  m{\'a}xima  del 50\%  de los tornillos con menor dimensi{\'o}n y
       la dimensi{\'o}n m{\'\i}nima del 5\%
             con mayor dimensi{\'o}n.
\item [d)] Si $Y$ denota la dimensi{\'o}n de los tornillos producidos en   otra f{\'a}brica, con la misma media y desviaci{\'o}n t{\'\i}pica que $X$,
dar un intervalo en el que tome valores la variable $Y$ con  una probabilidad
m{\'\i}nima de 0.99.
     \end{enumerate}
     
\subproblem

Para determinar el valor de k usaremos la siguiente característica de la función de densidad:
\begin{equation*}
    \int_{-\infty}^{+\infty} f(x) \,dx = 1
\end{equation*}

Teniendo esto en cuenta:
\begin{equation*}
    \int_{-\infty}^{+\infty} f(x) \,dx = \int_1^{10} \dfrac{k}{x^2} \,dx = \dfrac{k}{-x} \Big|^{10}_1 = \dfrac{-k}{10} + k = \dfrac{9k}{10} = 1 \Rightarrow k = \dfrac{10}{9}
\end{equation*}

Y finalmente tenemos que $k = \dfrac{10}{9}$. Para calcular la función de distribución debemos calcular la integral de la función de densidad, como ya la habíamos calculado antes, ahora solo tenemos que definirla. La definimos:
\begin{equation*}
    F(x) = \left\{ \begin{array}{lcc}
             0 &   si  & x < 1 \\
             \\ \dfrac{10/9}{-x} + \dfrac{10}{9} &  si & 1 \leq x \leq 10 \\
             \\ 1 &  si  & x > 10
             \end{array}
   \right.
\end{equation*}

\subproblem

Para calcular esta probabilidad usaremos la función de distribución:
\begin{equation*}
    P(2 \leq x \leq 5) = F(5) - F(2) = \dfrac{10/9}{-5} + \dfrac{10}{9} - \dfrac{10/9}{-2} -\dfrac{10}{9} = - \dfrac{2}{9} + \dfrac{5}{9} = \dfrac{3}{9} = \dfrac{1}{3}
\end{equation*}

\subproblem

En este apartado nos piden que calculemos el percentil 50 y el percentil 95, para calcularlos debemos resolver la ecuación $P(X \leq x_i) = \dfrac{r}{100}$ donde $r$ es el percentil que queremos calcular. Resolvemos estas ecuaciones:

-$P_{50}$:
\begin{gather*}
    P(X \leq x_i) = 0.5 ; \hspace{0.4cm} F(x_i) = 0.5 ; \hspace{0.4cm} \dfrac{10/9}{-x_i} + \dfrac{10}{9} = 0.5 ;  \\\hspace{0.4cm} 10-10x_i = -4.5x_i ; \hspace{0.4cm} x_i = \dfrac{20}{11} = 1.88
\end{gather*}

-$P_{95}$:
\begin{gather*}
    P(X \leq x_i) = 0.95 ; \hspace{0.4cm} F(x_i) = 0.95 ; \hspace{0.4cm}\dfrac{10/9}{-x_i} + \dfrac{10}{9} = 0.95 ; \hspace{0.4cm} \\ 10-10x_i = -8.55x_i ; \hspace{0.4cm}x_i = \dfrac{200}{29} \approx 6.897
\end{gather*}

\subproblem

Para este apartado usaremos la desigualdad de Chebychev de la siguiente forma:
\begin{equation*}
    P(E[Y] - \sigma_Y k \leq Y \leq E[Y] + \sigma_Y k) \geq 1-\dfrac{1}{k^2}
\end{equation*}

Primero calculamos k. Para ello usamos que la probabilidad debe ser como mínimo 0.99:
\begin{equation*}
    1-\dfrac{1}{k^2} = 0.99 \iff k^2 = 1/0.01 \iff k = \pm \sqrt{100} \iff k = \pm 10
\end{equation*}
Para que el intervalo $[E[Y] - \sigma_Y k, E[Y] + \sigma_Y k]$ sea correcto debemos tomar $k=10$. Ahora calcularemos la esperanza y la desviación típica de $Y$:
\begin{equation*}
    E[Y] = E[X] = \int_{-\infty}^{+\infty} xf(x) \,dx  = \dfrac{10}{9} \int_{1}^{10} \dfrac{1}{x} \,dx = \dfrac{10}{9} \cdot \ln x \Big|_1^{10} = \dfrac{10 \ln{10}}{9} \approx 2.558
\end{equation*}
\begin{gather*}
    \sigma_Y = \sigma_X = \sqrt{E[X^2]-E[X]^2} = \sqrt{\int_{-\infty}^{+\infty} x^2f(x) \,dx - E[X]^2} =\sqrt{ \dfrac{10}{9} \int_{1}^{10} 1 \,dx - E[X]^2}=\\
    =\sqrt{\dfrac{10}{9} \cdot x \Big|_1^{10}- E[X]^2} =\sqrt{10- \dfrac{100 \ln{10}^2}{81}} \approx 1.859
\end{gather*}

Sustituimos los valores calculados en la desigualdad:
\begin{equation*}
    P(E[Y] - \sigma_Y k \leq Y \leq E[Y] + \sigma_Y k) = P(-16.032 \leq Y \leq 21.148 )\geq 0.99
\end{equation*}

El intervalo sería $[-16.032,21.148]$ pero como $Y$ es una variable positiva el intervalo quedaría de la siguiente forma: $[0,21.148]$.
