\problem

\subproblem

Para determinar el valor de k usaremos la siguiente característica de la función de densidad:
\begin{equation*}
    \int_{-\infty}^{+\infty} f(x) \,dx = 1
\end{equation*}

Teniendo esto en cuenta:
\begin{equation*}
    \int_{-\infty}^{+\infty} f(x) \,dx = \int_1^{10} \dfrac{k}{x^2} \,dx = \dfrac{k}{-x} \Big|^{10}_1 = \dfrac{-k}{10} + k = \dfrac{9k}{10} = 1 \Rightarrow k = \dfrac{10}{9}
\end{equation*}

Y finalmente tenemos que $k = \dfrac{10}{9}$. Para calcular la función de distribución debemos calcular la integral de la función de densidad, como ya la habíamos calculado antes, ahora solo tenemos que definirla. La definimos:
\begin{equation*}
    F(x) = \left\{ \begin{array}{lcc}
             0 &   si  & x < 1 \\
             \\ \dfrac{10/9}{-x} \Big|^x_1 &  si & 1 \leq x \leq 10 \\
             \\ 1 &  si  & x > 10
             \end{array}
   \right.
\end{equation*}

\subproblem

Para calcular esta probabilidad usaremos la función de distribución:
\begin{equation*}
    P(2 \leq x \leq 5) = F(5) - F(2) = \dfrac{10/9}{-x} \Big|^5_1 - \dfrac{10/9}{-x} \Big|^2_1 = \dfrac{8}{9} - \dfrac{5}{9} = \dfrac{3}{9} = \dfrac{1}{3}
\end{equation*}

\subproblem

En este apartado nos piden que calculemos el percentil 50 y el percentil 95, para calcularlos debemos resolver la ecuación $P(X \leq x_i) = \dfrac{r}{100}$ donde $r$ es el percentil que queremos calcular. Resolvemos estas ecuaciones:

-$P_{50}$:
\begin{gather*}
    P(X \leq x_i) = 0.5 ; \hspace{0.4cm} F(x_i) = 0.5 ; \hspace{0.4cm} \dfrac{10/9}{-x} \Big|^{x_i}_1 = 0.5 ; \hspace{0.4cm} \dfrac{10/9}{-x_i} + \dfrac{10}{9} = 0.5 ; \hspace{0.4cm}\\ 10-10x_i = -4.5x_i ; \hspace{0.4cm} x_i = \dfrac{20}{11} = 1.88
\end{gather*}

-$P_{95}$:
\begin{gather*}
    P(X \leq x_i) = 0.95 ; \hspace{0.4cm} F(x_i) = 0.95 ; \hspace{0.4cm}\dfrac{10/9}{-x} \Big|^{x_i}_1 = 0.95 ; \hspace{0.4cm}\dfrac{10/9}{-x_i} + \dfrac{10}{9} = 0.95 ; \hspace{0.4cm} \\ 10-10x_i = -8.55x_i ; \hspace{0.4cm}x_i = \dfrac{200}{29} \approx 6.897
\end{gather*}
