\documentclass[11pt]{book}
\usepackage{amssymb}
\usepackage{amsmath}
\usepackage{array}
%\usepackage{fancyhea}
%\usepackage{supertab}
\usepackage{graphicx}
\usepackage[spanish]{babel}
%\usepackage{float}
\setlength{\textwidth}{17cm} \setlength{\textheight}{25cm}
\setlength{\oddsidemargin}{-0.5cm} \setlength{\evensidemargin}{-0.5cm}
\setlength{\topmargin}{-2.25cm} \pagestyle{empty}
\begin{document}
\centerline{\large \bf Relaci{\'o}n de Problemas 5: Variables aleatorias
unidimensionales}
\smallskip \centerline{\large \it Estad{\'\i}stica Descriptiva e Introducci{\'o}n a la
Probabilidad}

\centerline{Primer curso del Doble Grado en Ingenier\'ia Inform\'atica y
Matem{\'a}ticas} \hrulefill \vskip 0.5cm




\begin{enumerate}

\item Sea $X$ una variable aleatoria con funci{\'o}n masa de probabilidad
      $
      P(X = i)  = k  i ;\ \  i = 1, \ldots , 20.
      $
\begin{itemize}\item [$a)$]      Determinar el valor de $k$, la funci{\'o}n de distribuci{\'o}n y las siguientes
       probabilidades:
$$P(X=4) ,\  P(X<4) ,\  P(3 \leq X \leq 10) ,\
             P(3< X \leq 10) ,\  P(3<X<10).$$
\item [$b)$] Supongamos que un jugador gana $20$  monedas si al observar esta variable obtiene un
valor menor que $4$, gana  $24$ monedas si obtiene el valor $4$ y, en caso
contrario, pierde una moneda. Calcular la ganancia esperada del jugador y decir
si el juego le es favorable.
\end{itemize}
 \vskip 0.75cm \item Sea $X$ el n{\'u}mero de bolas blancas obtenidas al sacar dos de una urna con 10 bolas de las que 8 son blancas.
Calcular: \begin{itemize} \item [a)] Funci{\'o}n masa de probabilidad y  funci{\'o}n de
distribuci{\'o}n. \item [b)] Media, mediana y moda, dando la interpretaci{\'o}n de cada
una de estas medidas.\item [c)]Intervalo intercuart{\'\i}lico,  especificando  su
interpretaci{\'o}n.\end{itemize}

 \vskip 0.75cm \item  El n{\'u}mero de lanzamientos de una moneda hasta salir cara es  una variable aleatoria
con distribuci{\'o}n  $P(X = x)  =  2^{-x};\ \ x = 1, 2, \ldots$
     \begin{enumerate}
       \item [$a)$] Probar que la funci{\'o}n masa de probabilidad est{\'a} bien definida.
\item [$b)$] Calcular la probabilidad de que el n{\'u}mero de lanzamientos necesarios para salir cara est{\'e} entre $4$ y $10$.
\item [$c)$] Calcular los cuartiles y la moda de la distribuci{\'o}n, interpretando los valores.
\item [$d)$] Calcular la funci{\'o}n generatriz de momentos y, a partir de ella, el n{\'u}mero medio de lanzamientos necesarios para salir cara y la desviaci{\'o}n t{\'\i}pica.
     \end{enumerate}
\vskip 0.75cm \item Sea $X$ una variable aleatoria con funci{\'o}n de densidad
     $$
     f(x) = \left \{
     \begin{array}{lc}
       k_{1} (x+1) & 0 \leq x \leq 4 \\
       \\ k_{2} x^{2} & 4 <x \leq 6
     \end{array}
     \right.
     $$
Sabiendo que  $P(0 \leq X \leq 4) = 2/3$,  determinar $k_{1},\ k_{2}$, y
deducir su funci{\'o}n de distribuci{\'o}n.

\vskip 0.75cm  \item  La dimensi{\'o}n en cent{\'\i}metros  de  los tornillos  que salen
de cierta f{\'a}brica es una  variable aleatoria, $X$, con funci{\'o}n de densidad
     $$
     f(x) = \displaystyle{\frac{k}{x^{2}}},\ \    1 \leq x \leq 10.
     $$
     \begin{enumerate}
       \item [$a)$]Determinar el valor de $k$, y obtener la funci{\'o}n de distribuci{\'o}n.
       \item [$b)$] Hallar la probabilidad de que la dimensi{\'o}n de un tornillo
             est{\'e}  entre 2 y 5 cm.
       \item [$c)$] Determinar la  dimensi{\'o}n  m{\'a}xima  del 50\%  de los tornillos con menor dimensi{\'o}n y
       la dimensi{\'o}n m{\'\i}nima del 5\%
             con mayor dimensi{\'o}n.
\item [d)] Si $Y$ denota la dimensi{\'o}n de los tornillos producidos en   otra f{\'a}brica, con la misma media y desviaci{\'o}n t{\'\i}pica que $X$,
dar un intervalo en el que tome valores la variable $Y$ con  una probabilidad
m{\'\i}nima de 0.99.
     \end{enumerate}

\vskip 0.75cm  \item  Sea $X$  una variable aleatoria  con funci{\'o}n de densidad
     $$
     f(x) = \left \{
     \begin{array}{lc}
       \displaystyle{\frac{2x-1}{10}} & 1<x \leq 2 \\
       \ & \ \\
       \displaystyle{0\mbox{.}4} & 4 < x \leq 6.
       \end{array}
     \right.
     $$
    \begin{enumerate}\item [$a)$] Calcular \ $P(1\mbox{.}5 < X \leq 2) ,\  P(2\mbox{.}5<X \leq 3\mbox{.}5) ,\
    P(4\mbox{.}5 \leq X < 5\mbox{.}5) ,\  P(1\mbox{.}2 < X \leq
     5\mbox{.}2)$.
     \item [$b)$] Dar la expresi{\'o}n general de los momentos no centrados y deducir el valor  medio de $X$.
     \item [$c)$] Calcular la funci{\'o}n generatriz de momentos de $X$.
     \end{enumerate}

\vskip 0.75cm   \item  Con objeto de establecer un plan de producci{\'o}n,  una
empresa  ha estimado que la demanda de sus clientes, en miles de unidades del
producto,  se comporta  semanalmente  con arreglo a una ley de probabilidad
dada por la funci{\'o}n de densidad:
     $$
     f(x) =
       \displaystyle{\frac{3}{4} (2x - x^{2})}, \  \ 0 \leq x \leq 2.
     $$
\begin{enumerate}\item [$a)$] ?`Qu{\'e}  cantidad  deber{\'a} tener dispuesta a la venta al comienzo  de  cada  semana para poder
satisfacer plenamente la demanda con probabilidad 0.5?
     \item [$b)$] Pasado cierto tiempo, se observa  que la demanda ha crecido,   estim{\'a}ndose que en
     ese momento se
     distribuye seg{\'u}n la  funci{\'o}n de densidad:
      $$
     f(y) =  \displaystyle{\frac{3}{4} (4y - y^{2}-3)},  \ \ 1 \leq y \leq 3.
     $$
Los empresarios  sospechan que este crecimiento no ha afectado  a la dispersi{\'o}n
de la demanda, ?`es cierta esta sospecha?
\end{enumerate}

\vskip 0.75cm \item  Calcular las funciones masa de probabilidad de las
variables $Y = X + 2$ y $Z = X^{2}$, siendo $X$ una variable aleatoria  con
distribuci{\'o}n:
  $$
P(X=-2)= \displaystyle\frac{1}{5}, \hskip 0.5cm P(X=-1)=
\displaystyle\frac{1}{10}, \hskip 0.5cm P(X=0)= \displaystyle\frac{1}{5},
\hskip 0.5cm P(X=1)= \displaystyle\frac{2}{5}, \hskip 0.5cm P(X=2)=
\displaystyle\frac{1}{10}\cdot
$$
?`C{\'o}mo afecta el cambio de $X$ a $Y$ en el coeficiente de variaci{\'o}n?

\vskip 0.75cm \item  Calcular las funciones de densidad de  las variables $Y=2
X + 3$ \  y \  $Z=|X |$, siendo  $X$ una variable continua  con funci{\'o}n de
densidad
     $$
     f_X(x) =  \frac{1}{4}, \ \  -2 < x < 2.
     $$


\vskip 0.75cm \item Si $X$ es una variable aleatoria con funci{\'o}n de densidad
$$f(x)=\frac{e^{-|x|}}{2}, \ \ -\infty < x < \infty,$$ hallar su funci{\'o}n de distribuci{\'o}n
y  las probabilidad de cada uno de los
      siguientes sucesos:
      \begin{enumerate}
        \item [$a)$]$\{|X|\leq 2\}$.
        \item [$b)$]$\{|X|\leq 2\ \ \ \mbox{{\'o}} \ \ X \geq 0\}.$
        \item [$c)$] $\{|X|\leq 2\ \ \ \mbox{y} \ \ X \leq -1\}.$
       % \item [d)] $\{|X|+|X-3|\leq 3\}.$
        \item [$d)$] $\{X^3-X^2-X-2 \leq 0 \}.$
        %\item $\{\exp(\sin \pi X)\geq 1\}.$
        \item [$e)$] $\{ X \mbox{ es irracional}\}.$
      \end{enumerate}


\vskip 0.75cm \item Sea $X$ una variable aleatoria  con funci{\'o}n de densidad
$$f(x)=1, \ \ 0 \leq x \leq 1.$$ Encontrar
      la distribuci{\'o}n  de las variables:
      \begin{itemize}
\item [$a)$]  $Y=\displaystyle\frac{X}{1+X}\cdot$
\item [$b)$]  $Z=\left\{\begin{array}{ll}-1, & X<3/4\\ 0, & X=3/4\\ 1, & X>3/4.\end{array}\right.$
\end{itemize}

\vskip 0.75cm \item  Sea $X$ una variable aleatoria sim{\'e}trica con respecto al
punto 2, y con coeficiente de variaci{\'o}n 1. ?`Qu{\'e} puede decirse acerca de las
siguientes probabilidades?:
\begin{itemize}
\item $P(-8<X<12)$
\item $P(-6<X<10).$
\end{itemize}
\end{enumerate}
\end{document}
