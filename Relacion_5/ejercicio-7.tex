\problem 

Con objeto de establecer un plan de producci{\'o}n,  una
empresa  ha estimado que la demanda de sus clientes, en miles de unidades del
producto,  se comporta  semanalmente  con arreglo a una ley de probabilidad
dada por la funci{\'o}n de densidad:
$$
f(x) =
\displaystyle{\frac{3}{4} (2x - x^{2})}, \  \ 0 \leq x \leq 2.
$$
\begin{enumerate}\item [$a)$] ?`Qu{\'e}  cantidad  deber{\'a} tener dispuesta a la venta al comienzo  de  cada  semana para poder
	satisfacer plenamente la demanda con probabilidad 0.5?
	\item [$b)$] Pasado cierto tiempo, se observa  que la demanda ha crecido,   estim{\'a}ndose que en
	ese momento se
	distribuye seg{\'u}n la  funci{\'o}n de densidad:
	$$
	f(y) =  \displaystyle{\frac{3}{4} (4y - y^{2}-3)},  \ \ 1 \leq y \leq 3.
	$$
	Los empresarios  sospechan que este crecimiento no ha afectado  a la dispersi{\'o}n
	de la demanda, ?`es cierta esta sospecha?
\end{enumerate}

\subproblem 
Se nos pide calcular el percentil 50, es decir, la mediana de la distribución de probabilidad. Por tanto, tenemos que calcular la función de distribución haciendo la integral de la función de densidad.

$$\int \dfrac{3}{4}(2x-x^2) = \dfrac{3x^2}{4} - \dfrac{x^3}{4}$$

$$F(x) = 0.5;\quad \dfrac{x^2}{4} - \dfrac{x^3}{4} = 0.5$$

$$3x^2 - x^3 - 2 = 0$$

1 es solución de la ecuación y se encuentra entre 0 y 2, las cotas de los datos, por tanto, se necesitarán preparar 1000 unidades del producto.

\subproblem 
Para calcular las varianzas tendremos que hacer los siguientes cálculos:

$$E[X] = \int_{-\infty}^{\infty} x\cdot f(x)dx = \int_{-\infty}^0 0dx + \dfrac{3}{4} \int_0^2 2x^2 - x^3dx + \int_2^{\infty} 0dx = \dfrac{3\cdot2x^3}{4\cdot3} \biggr\rvert_0^2 - \dfrac{3x^4}{4\cdot4}\biggr\rvert_0^2 = 1$$

$$E[Y] = \int_{-\infty}^{\infty} y\cdot f(y)dy = \int_{-\infty}^1 0dy + \dfrac{3}{4} \int_1^3 4y^2 - y^3 -3y\ dy + \int_3^{\infty} 0dy = $$ 
$$= \dfrac{3\cdot4y^3}{4\cdot3} \biggr\rvert_1^3 - \dfrac{3y^4}{4\cdot4}\biggr\rvert_1^3 - \dfrac{3\cdot3y^2}{4\cdot2}\biggr\rvert_1^3= 2$$


$$m_2[X] = \int_{-\infty}^{\infty} x^2\cdot f(x)dx = \int_{-\infty}^0 0dx + \dfrac{3}{4} \int_0^2 2x^3 - x^4dx + \int_2^{\infty} 0dx = \dfrac{3\cdot2x^3}{4\cdot4} \biggr\rvert_0^2 - \dfrac{3x^4}{4\cdot5}\biggr\rvert_0^2 = 1.2$$

$$m_2[Y] = \int_{-\infty}^{\infty} y^2\cdot f(y)dy = \int_{-\infty}^1 0dy + \dfrac{3}{4} \int_1^3 4y^3 - y^4 -3y^2\ dy + \int_3^{\infty} 0dy = $$ 
$$= \dfrac{3y^4}{4} \biggr\rvert_1^3 - \dfrac{3y^5}{4\cdot5}\biggr\rvert_1^3 - \dfrac{3y^3}{4}\biggr\rvert_1^3= \dfrac{21}{5}$$

$$\mbox{Var}[X] = m_2[X] - E[X]^2 = 0.2 \implies \sigma_x = \dfrac{\sqrt{5}}{5}$$
$$\mbox{Var}[Y] = m_2[Y] - E[Y]^2 = 0.2 \implies \sigma_y = \dfrac{\sqrt{5}}{5}$$

$$C.V.[X] = \dfrac{\sigma_x}{E[X]} = \dfrac{\frac{\sqrt{5}}{5}}{1} = \dfrac{\sqrt{5}}{5}$$
$$C.V.[Y] = \dfrac{\sigma_y}{E[Y]} = \dfrac{\frac{\sqrt{5}}{5}}{2} = \dfrac{\sqrt{5}}{10}$$

Por tanto el crecimiento sí ha afectado a la dispersión, de hecho, ha reducido a la mitad el coeficiente de variación de Pearson.

