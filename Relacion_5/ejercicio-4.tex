\problem
Sea $X$ una variable aleatoria con funci{\'o}n de densidad
$$
f(x) = \left \{
\begin{array}{lc}
k_{1} (x+1) & 0 \leq x \leq 4 \\
\\ k_{2} x^{2} & 4 <x \leq 6
\end{array}
\right.
$$
Sabiendo que  $P(0 \leq X \leq 4) = 2/3$,  determinar $k_{1},\ k_{2}$, y
deducir su funci{\'o}n de distribuci{\'o}n.

Nótese que la variable aleatoria es continua y toma valores entre 0 y 6. En el desarrollo de este ejercicio, se presupondrá que la función de masa dada es correcta. 

En primer lugar, calculemos la función de distribución de la variable aleatoria. Aplicamos la siguiente expresión: $F(x) = \int_{-\infty}^x f(t) dt$. 

Si $0 \leq x \leq 4$, tenemos que: 

$$F(x) = \int_{-\infty}^x f(t) dt = \int_{-\infty}^0 0 dt + \int_0^x k_1(t+1)dt = k_1(\frac{·x^2}{2} + x)$$ 

Si $4 < x \leq 6$, tenemos que:

\begin{equation*}
\begin{split}
F(x) & = \int_{-\infty}^x f(t) dt = \int_{-\infty}^0 0 dt + \int_0^4 k_1(t+1)dt + \int_4^x k_2·t^2 dt \\
& = k_1(\frac{4^2}{2} + 4) + k_2\frac{x^3}{3} 
\end{split}
\end{equation*}

Sabemos que $P(0\leq X \leq 4) = F(4) - F(0^-)$. Como la variable es continua, tenemos que $F(0^-) = F(0)$. Deducimos entonces que: 

$$P(0 \leq X \leq 4) = F(4) - F(0^-) = F(4) - F(0) = k_1(\frac{4^2}{2} + 4) = 2/3 \Rightarrow k_1 = \frac{1}{18}$$

Por otra parte, para que se verifique que se trate, en efecto, de una función de distribución, se ha de verificar que: 

$$F(+\infty) = k_2\frac{6^3}{3} + \frac{2}{3} = 1 \Rightarrow k_2 = \frac{1}{216}$$

Por tanto, la función de distribución viene dado por la siguiente expresión: 

$$
F(x) = \left \{
\begin{array}{lc}
\frac{1}{18} (x+1) & 0 \leq x \leq 4 \\
\\ \frac{x^{2}}{216} & 4 <x \leq 6
\end{array}
\right.
$$
