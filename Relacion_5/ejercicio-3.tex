\problem 
Sabemos que se trata de una variable aleatoria discreta. \\
Para probar que la función masa de probabilidad está bien definida, comprobamos que cumple: \\
\subproblem
1. $P(X=x_i)\geq 0$ $\forall i\in N$ $\Rightarrow$ $P[X=x]=\frac{1}{2^x}$ que siempre va a ser positiva pues nunca se anula; y $2^x$ está siempre en el intervalo $(0,+\infty)$, por lo que la fracción nunca va a ser negativa. \\
2. $\sum_{i=1}^{\infty}P[X=x_i]=\sum_{i=1}^{\infty}\frac{1}{2^x} = \frac{1}{2} + \frac{1}{2} + \frac{1}{8} + ... $ Esta es una serie geométrica de razón $\frac{1}{2}$ $\Rightarrow \frac{1}{2} < 1 \Rightarrow$ la serie converge. Su suma viene dada por $\sum_{i}^n 2^{-n} = \frac{\frac{1}{2}}{1-\frac{1}{2}}=1$ \\ \\
\subproblem
$P(4\leq x \leq 10) = \sum^{10}_{i=4} p_i = \sum^{10}_{i=4} 2^{-x_i} = \frac{1}{2^4}+\frac{1}{2^5}+ \frac{1}{2^6} + \frac{1}{2^7} + \frac{1}{2^8} + \frac{1}{2^9}+ \frac{1}{2^{10}} = \frac{127}{1024} \simeq 0'124$ \\ \\
\subproblem
$Q_1: P(X\leq x_i) = \frac{25}{100} = 0'25 \Rightarrow \sum^{x_i}_{i=1} 2^{-x_i} = 0'25 < 0'5 = 2^{-1} \Rightarrow Q_1 = 1$ \\
$Q_2: P(X \leq x_i) = \frac{50}{100} = 0'5 \Rightarrow 2^{-1} = 0'5 \Rightarrow Q_2 = [1,2)$ \\
$Q_3: P(X \leq x_i) = \frac{75}{100 = 0'75} \Rightarrow \sum^2_{i=1} 2^{-x_i} = 0'5 + 0'25 = 0'75 \Rightarrow Q_3 = [2,3)$ \\
$M_o$: valor máximo. Cuanto mayor es $x_i$, menor es $\frac{1}{2^x_i}$. Por lo tanto, el valor máximo va a ser el primero, cuando $x_i = 1 \Rightarrow Mo = 1$ \\ \\
\subproblem
Función generatriz de momentos: $M_x(t) = \sum^\infty_{x=1} e^{tx}*P[X=x_i] = \sum^\infty_{x=1} e^{tx}*\frac{1}{2^x} = \sum^\infty_{x=1} (\frac{e^t}{2})^x = \frac{\frac{e^t}{2}}{1-\frac{e^t}{2}}$. Esto es una serie geométrica que converge si $\frac{e^t}{2} < 1 \Rightarrow e^t < 2 \Rightarrow t < log(2)$. Por lo tanto, converge en un entorno de cero. \\
Esperanza: $E[X] = M^{'} _x(t) = \frac{(\frac{e^t}{2})^{'}(1-\frac{e^t}{2})-(\frac{e^t}{2})(1-\frac{e^t}{2})^{'}}{(1-\frac{e^t}{2})^2} = \frac{\frac{e^t}{2}}{1 - e^t + \frac{e^t}{4}}$. Sustituimos t=0: $\frac{\frac{1}{2}}{\frac{1}{4}} = 2 = E[X]$ \\
Desviación típica: $E[X^2] = M^{''}_x(t) = \frac{\frac{e^t}{2}-\frac{e^{3t}}{8}}{1-2e^t+\frac{3e^{2t}}{2}-\frac{e^{3t}}{2}+\frac{e^{4t}}{16}}$. Sustituimos t=0: $\frac{\frac{6}{16}}{\frac{1}{16}} = 6$ \\
Varianza: $Var[X] = E[X^2]-E[X]^2 = 6-2^2 = 2 \Rightarrow \sigma_x = \sqrt{2} \simeq 1'414$

